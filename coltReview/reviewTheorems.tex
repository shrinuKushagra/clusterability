\documentclass{article}
\usepackage{amsthm, amsmath, amssymb}
\usepackage{geometry}

\newtheorem{theorem}{Theorem}
\newtheorem{definition}{Definition}

\newcommand{\mb}{\mathbb}
\newcommand{\mc}{\mathcal}
\DeclareMathOperator{\vcdim}{VC-Dim}
\DeclareMathOperator{\vol}{vol}

\begin{document}
The theorem below is a classical result relating to $\epsilon$-approximations in VC-Theory.
\begin{theorem}[Vapnik and Chervonenkis]
\label{theorem:vceapprox}
Let $X$ be a domain set and $D$ a probability distribution over $X$. Let $H$ be a class of subsets of $X$ of finite VC-dimension $d$. Let $\epsilon, \delta \in (0,1)$. Let $S \subseteq X$ be picked i.i.d according to $D$ of size $m$. If $m > \frac{c}{\epsilon^2}(d\log \frac{d}{\epsilon}+\log\frac{1}{\delta})$, then  with probability $1-\delta$ over the choice of $S$, we have that $\forall h \in H$
$$\bigg|\frac{|h\cap S|}{|S|} - P(h)\bigg| < \epsilon$$
\end{theorem}

We will use this result to theoretically justify our sparseness assumption. We will show that if a set of points are generated uniformly at random in a ball in $\mb R^d$ then with high probability they will satisfy the sparse condition. 

\begin{theorem}[Uniform sets are sparse]
\label{theorem:sparse}
Let $X$ be a ball of radius $R$ in $\mb R^d$. Let $\mc S \subseteq X$ be the clustering instance and let $C$ be a clustering of $S$ which satisfies $\alpha$-center proximity (or $\lambda$-center separation). Given parameters $\epsilon, \delta \in (0,1)$. Let $N \subseteq X$ be picked i.i.d according to the uniform distribution of size $m$. If 
\begin{itemize}
\item $m > O(\frac{d}{\epsilon^2}\log \frac{d}{\epsilon}+\frac{d}{\epsilon^2}\log\frac{1}{\delta})$
\item $\frac{m(C)}{m}> \Big(\frac{r(C)}{R}\eta\Big)^d + \epsilon$ 
\end{itemize}
then  with probability $1-\delta$, $C$ satisfies $(\alpha, \eta)$-center proximity (or $(\lambda, \eta)$-center separation).
\end{theorem}
\begin{proof}
Let $H = \{B \text{ is a ball }: B \subseteq X\}$. Observe that $\vcdim(H) = d+1$. Define $\gamma := \frac{r(C)}{R}$. Also, since $U$ is the uniform distribution, $P(B) = \frac{\vol(B)}{\vol(X)} = \gamma^d$. Now, using the result for $\epsilon$-approximation (Thm. \ref{theorem:vceapprox}), completes the proof of this theorem.
\end{proof}
\end{document}