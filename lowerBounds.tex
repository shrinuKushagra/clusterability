\documentclass[11pt]{article}
\usepackage[paper]{nickstyle}
\usepackage{color}
\usepackage{hyperref}
\usepackage{amssymb, amsmath}
\usepackage{enumitem}

\newcommand{\mc}{\mathcal}
\setlength{\parindent}{24pt}
\renewcommand{\bar}[1]{\overline{#1}}

\newtheorem{fact}[theorem]{Fact}
\newtheorem{conj}[theorem]{Conjecture}
\newtheorem{problem}[theorem]{Problem}
\newcommand{\ConjName}[1]{\label{con:#1}}
\newcommand{\Conj}[1]{Conjecture~\ref{con:#1}}
\newcommand{\ProblemName}[1]{\label{prob:#1}}
\newcommand{\Problem}[1]{Problem~\ref{prob:#1}}
\renewcommand{\dot}{\bullet}
\newcommand{\Tr}{\operatorname{tr}}
\newcommand{\eps}{\epsilon}

\newcommand{\lmax}{\lambda_\mathrm{max}}
\newcommand{\lmin}{\lambda_\mathrm{min}}
\newcommand{\ufinal}{u_\mathrm{final}}
\newcommand{\lfinal}{l_\mathrm{final}}
\newcommand{\umax}{u_\mathrm{max}}

\newcommand{\Symraw}{\mathbb{S}}
\newcommand{\Sym}[1][]{\Symraw^{\ifthenelse{\equal{#1}{}}{m}{#1}}}
\newcommand{\Psd}[1][]{\Symraw_+^{\ifthenelse{\equal{#1}{}}{m}{#1}}}
\newcommand{\Reals}{\mathbb{R}}
\newcommand{\iprod}[2]{\langle #1, #2 \rangle}
\newcommand{\paren}[2][]{#1({#2}#1)}
\newcommand{\qform}[2]{\transp{#2}#1#2}
\newcommand{\transp}[1]{#1^T}



% Simple (outer) environment for algorithms
\newenvironment{outer_alg}{
    \begin{list}{}{
        \setlength{\itemsep}{2pt}
        \setlength{\parsep}{0pt}
        \setlength{\parskip}{0pt}
        \setlength{\topsep}{1pt}
        \setlength{\leftmargin}{5pt}
    }
}
{
    \end{list}
}

% Simple environments for algorithms
\newenvironment{alg}{
    \begin{list}{}{
        \setlength{\itemsep}{2pt}
        \setlength{\parsep}{0pt}
        \setlength{\parskip}{0pt}
        \setlength{\topsep}{1pt}
    }
}
{
    \end{list}
}

%%%%% Title %%%%%
\title{\LARGE Lower bound}
\author{}


%%%%% Document Body %%%%%
\begin{document}
\maketitle

\section{Lower bounds for $(\alpha, \eta)$-center proximity}

\begin{definition}[$(\alpha, \eta)$-center proximity]
Given a clustering instance $(\mc X, d)$, the number of clusters $k$ and $\mc S \subseteq \mc X$. We say that a clustering $\mc C = \{C_1, \ldots, C_k\}$ of $\mc S$ induced by centers $c_1, \ldots, c_k$ has $(\alpha, \eta)$-center proximity w.r.t $\mc X, \mc S$ and $k$ if the following holds.

\begin{itemize}[nolistsep, noitemsep]
\label{defn:alphacpnoise}	

\item[$\diamond$] {\bf $\alpha$-centre proximity}: For all $x \in C_i$ and $i\neq j$, $\thinspace\alpha d(x, c_i) < d(x, c_j)$
\item[$\diamond$]{\bf $\eta$-sparse noise}: For any ball $B$ such that $c(B)\in \mathcal{X}$, if $r(B)\leq \eta\max\limits_{i} \thinspace r(C_i)$, then $|B\cap (\mc X\setminus \mc S)| < \min\limits_{i} |C_i|/2$
\end{itemize}
\end{definition}

\noindent A set $\mc S \subseteq \mc X$ is {\it $(\alpha, \eta)$-center} {\it $t$-min nice} if there exist a clustering $\mc C=\{C_1,\ldots,C_k\}$ of $\mc S$ which satisfies $(\alpha, \eta)$-center proximity and $\min\limits_{i} \lvert C_i\rvert = t$. Note that given $(\mc X, d), k$ and $t$ there can be several such $\mc S$ and several clusterings $\mc C$ of $\mc S$ which satisfy these conditions.

\subsection{Goal}
We are given a clustering instance $(\mc X, d)$, the number of clusters $k$ and a parameter $t$. Our goal is to output a hierarchical clustering tree of $\mc X$ which has the following property. For every $\mc S \subseteq \mc X$ which is $(\alpha, \eta)$-center $t$-min nice, there exists a pruning $\mc P'$ of the tree such that $\mc C'$ (the clustering of the set $\mc X$ corresponding to the pruning) respects $\mc C$ (any clustering of $\mc S$ which satisfies $(\alpha, \eta)$-center proximity). 

In the previous section we proposed a hierarchical clustering algorithm (Alg.\ref{alg:alphacp}) which acheives the desired goal for $\alpha > 2 + \sqrt{7}$ and $\eta \ge 1$. In this section, we want to show some negative results. Specifically, we will show that for $\alpha \le 2 + \sqrt{3}$ no algorithm exists which can acheive the above mentioned goal.

\begin{theorem}
Let $\mc X \subset \mathbf{R}$. 
\end{theorem}

%%%%%%%%%%%%%%%%%%%%%%%%%%%%%%%%%%%%%%%%%%%%%%%%%%%%%%%%%%%%%%%%%%%


























\end{document}
