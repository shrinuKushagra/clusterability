\documentclass[11pt]{article}


%%%%% Packages %%%%%
\usepackage[paper]{nickstyle}
\usepackage{color}
\usepackage{hyperref}
\usepackage{amssymb, amsmath}



%%%%% Macros %%%%%
\setlength{\parindent}{24pt}
\renewcommand{\bar}[1]{\overline{#1}}

\newtheorem{fact}[theorem]{Fact}
\newtheorem{conj}[theorem]{Conjecture}
\newtheorem{problem}[theorem]{Problem}
\newcommand{\ConjName}[1]{\label{con:#1}}
\newcommand{\Conj}[1]{Conjecture~\ref{con:#1}}
\newcommand{\ProblemName}[1]{\label{prob:#1}}
\newcommand{\Problem}[1]{Problem~\ref{prob:#1}}
\renewcommand{\dot}{\bullet}
\newcommand{\Tr}{\operatorname{tr}}
\newcommand{\eps}{\epsilon}

\newcommand{\lmax}{\lambda_\mathrm{max}}
\newcommand{\lmin}{\lambda_\mathrm{min}}
\newcommand{\ufinal}{u_\mathrm{final}}
\newcommand{\lfinal}{l_\mathrm{final}}
\newcommand{\umax}{u_\mathrm{max}}

\newcommand{\Symraw}{\mathbb{S}}
\newcommand{\Sym}[1][]{\Symraw^{\ifthenelse{\equal{#1}{}}{m}{#1}}}
\newcommand{\Psd}[1][]{\Symraw_+^{\ifthenelse{\equal{#1}{}}{m}{#1}}}
\newcommand{\Reals}{\mathbb{R}}
\newcommand{\iprod}[2]{\langle #1, #2 \rangle}
\newcommand{\paren}[2][]{#1({#2}#1)}
\newcommand{\qform}[2]{\transp{#2}#1#2}
\newcommand{\transp}[1]{#1^T}



% Simple (outer) environment for algorithms
\newenvironment{outer_alg}{
    \begin{list}{}{
        \setlength{\itemsep}{2pt}
        \setlength{\parsep}{0pt}
        \setlength{\parskip}{0pt}
        \setlength{\topsep}{1pt}
        \setlength{\leftmargin}{5pt}
    }
}
{
    \end{list}
}

% Simple environments for algorithms
\newenvironment{alg}{
    \begin{list}{}{
        \setlength{\itemsep}{2pt}
        \setlength{\parsep}{0pt}
        \setlength{\parskip}{0pt}
        \setlength{\topsep}{1pt}
    }
}
{
    \end{list}
}




%%%%% Title %%%%%
\title{\LARGE Clusterability}
\author{}


%%%%% Document Body %%%%%
\begin{document}
\maketitle

\section{Problem Statement}

\subsection{Notation}
For any set $B\subset X$, we denote $c(B)$ as the center of $B$ which is defined as the average of points in $B$. Radius of the set $B$ is defined as $r(B)=\max_{x\in B} |x-c(B)|$. For a given partitioning of set $\mathcal{X}$

\begin{definition}[Niceness assumption]
% Adding C_{k+1} as garbage collector?
Given a set $\mathcal{X}$, we say that a partition of $\mathcal{X}$, $P=\{P_1,\ldots,P_k\}$ is $(\lambda,\nu)$-nice if the following conditions hold. There exist sets $B=B_1,\ldots,B_k\subset \mathcal{X}$ such that for every $i\in[k]$, there exists $j_i\in[k]$ such that $B_i\subset P_{j_i}$ and
\begin{itemize}
\item{\bf{Separation}:} For all $i,j\in[k], |c(B_i)-c(B_j)|\geq \nu\cdot\max\{r(B_i),r(B_j)\}$
\item{\bf{Sparse Noise}}: For any ball $B\subset \mathcal{X}$ for which $r(B)\leq \lambda \cdot max_{i\in[k]} r(B_i)$, $|B\cap \{X \backslash \cup_{i\in[k]} B_i\}\leq \min_{i\in[k]}|B_i|$.
\end{itemize}
\end{definition}

Goal: Gievn the set $\mathcal{X}$ and the value $k$, our goal is to design an algorithm that do $k$-clustering $C=\{C_1,\ldots,C_k\}$ on set $\mathcal{X}$ such that  for any $(\lambda,\nu)$-nice partitioning $P$ with slusters $B_1,\ldots,B_k$ we have that $C|B={B_1,\ldots,B_k}$

\section{Algorithm}


\subsection{Based on the knowledge of $\min |B_i|$}
The algorithm gets as input $\mathcal{X}$ and the number of points in the smallest cluster $\min |B_i| = t$ (say). This algorithm works in two phases.





\end{document}
