\documentclass[11pt]{article}
\usepackage[paper]{nickstyle}
\usepackage{color}
\usepackage{hyperref}
\usepackage{amssymb, amsmath}

\newcommand{\mc}{\mathcal}
\setlength{\parindent}{24pt}
\renewcommand{\bar}[1]{\overline{#1}}

\newtheorem{fact}[theorem]{Fact}
\newtheorem{conj}[theorem]{Conjecture}
\newtheorem{problem}[theorem]{Problem}
\newcommand{\ConjName}[1]{\label{con:#1}}
\newcommand{\Conj}[1]{Conjecture~\ref{con:#1}}
\newcommand{\ProblemName}[1]{\label{prob:#1}}
\newcommand{\Problem}[1]{Problem~\ref{prob:#1}}
\renewcommand{\dot}{\bullet}
\newcommand{\Tr}{\operatorname{tr}}
\newcommand{\eps}{\epsilon}

\newcommand{\lmax}{\lambda_\mathrm{max}}
\newcommand{\lmin}{\lambda_\mathrm{min}}
\newcommand{\ufinal}{u_\mathrm{final}}
\newcommand{\lfinal}{l_\mathrm{final}}
\newcommand{\umax}{u_\mathrm{max}}

\newcommand{\Symraw}{\mathbb{S}}
\newcommand{\Sym}[1][]{\Symraw^{\ifthenelse{\equal{#1}{}}{m}{#1}}}
\newcommand{\Psd}[1][]{\Symraw_+^{\ifthenelse{\equal{#1}{}}{m}{#1}}}
\newcommand{\Reals}{\mathbb{R}}
\newcommand{\iprod}[2]{\langle #1, #2 \rangle}
\newcommand{\paren}[2][]{#1({#2}#1)}
\newcommand{\qform}[2]{\transp{#2}#1#2}
\newcommand{\transp}[1]{#1^T}



% Simple (outer) environment for algorithms
\newenvironment{outer_alg}{
    \begin{list}{}{
        \setlength{\itemsep}{2pt}
        \setlength{\parsep}{0pt}
        \setlength{\parskip}{0pt}
        \setlength{\topsep}{1pt}
        \setlength{\leftmargin}{5pt}
    }
}
{
    \end{list}
}

% Simple environments for algorithms
\newenvironment{alg}{
    \begin{list}{}{
        \setlength{\itemsep}{2pt}
        \setlength{\parsep}{0pt}
        \setlength{\parskip}{0pt}
        \setlength{\topsep}{1pt}
    }
}
{
    \end{list}
}




%%%%% Title %%%%%
\title{\LARGE Clusterability with structural noise}
\author{}


%%%%% Document Body %%%%%
\begin{document}
\maketitle

\section{Introduction}
\begin{itemize}
\item Nika and Shai ICML'14
\end{itemize}

%%%%%%%%%%%%%%%%%%%%%%%%%%%%%%%%%%%%%%%%%%%%%%%%%%%%%%%%%%%%%%%%%%%%%%%%%%%%%%%%%%%%%%%%%%%%%%%%%%%%%%%%%%%%%%%
\section{Problem Statement}



%%%%%%%%%%%%%%%%%%%%%%%%%%%%%%%%%%%%%%%%%%%%%%%%%%%%%%%%
\subsection{Notation}
For any set $B\subset \mc X$, we denote $c(B)$ as the center of $B$ which is defined as the average of points in $B$. Radius of the set $B$ is defined as $r(B)=\max_{x\in B} |x-c(B)|$. 

A $k$-clustering of the set $\mc X$ is defined as the set $\mc C = \{C_1,\ldots,C_k\}$ such that $C_i \subseteq \mc X$, $\cup C_i = \mc X$ and $C_i \cap C_j = \phi$. For any $x \in \mc X$, if $x \in C_i$, we say that $x$ lies in the $i^{th}$ cluster. Sometimes, we also denote the clustering by $\mc C = \{c_1,\ldots,c_k\}$ where $c_i \in \mc X$. For any $x \in \mc X$, $x$ lies in the cluster given by $\argmin |x-c_i|$.


%%%%%%%%%%%%%%%%%%%%%%%%%%%%%%%%%%%%%%%%%%%%%%%%%%%%%%%%
\subsection{Niceness assumption}
\begin{definition}[Niceness assumption]
% Adding C_{k+1} as garbage collector?
Given a set $\mc X$, we say that $\mc X$ is $(\lambda,\nu)$-nice if there exists a clustering of $\mc X$, $\mc C=\{C_1,\ldots,C_k\}$ which satisfies the following conditions. There exists balls $\mc B=\{B_1,\ldots,B_k\}\subset \mathcal{X}$ such that for every $i\in[k]$, there exists $j(i)\in[k]$ such that $B_i\subset C_{j(i)}$ and
\begin{itemize}
\item{\bf{Separation}:} For all $i,j\in[k], |c(B_i)-c(B_j)| > \nu\cdot\max_t \thinspace r(B_t)$
\item{\bf{Sparse Noise}}: For any ball $B$ such that $c(B)\in \mathcal{X}$, if $r(B)\leq \lambda \cdot \max \thinspace r(B_i)$, then $|B\cap \{X \backslash \cup_{i\in[k]} B_i\}| < \min_i |B_i|$.
\end{itemize}
\label{defn:niceness}
\end{definition}

\noindent {\bf Goal}: Given the set $\mc X$ and the value $k$, our goal is to design an algorithm that outputs a $k$-clustering $\mc C' =\{C_1',\ldots,C_k'\}$ on set $\mc X$ such that  for any $(\lambda,\nu)$-nice clustering $\mc C$ with sets $\mc B = \{B_1,\ldots,B_k\}$, the induced clustering of $\mc C'$ on $\mc B$, i.e., $\mc C'|_\mc B = \{B_1,\ldots,B_k\}$. 

\begin{lemma}
\label{lemma:chknice}
Given a $(\lambda,\nu)$-nice set $\mc X$. Given a $k$-clustering $\mc C = \{C_1,\ldots,C_k\}$ of $\mc X$ and corresponding balls $\mc B = \{B_1,\ldots,B_k\}$, i.e. $B_i \subseteq C_i$, we can check if $\mc C$ is nice in time poynomial in $|\mc X|$ and $k$.
\end{lemma}
\begin{proof}
We can check if the balls $B_i$ satisfy the separation condition in time $O(k^2)$. To check if they satisfy the sparse noise condition, it is easy to see that it takes time $O(n^2)$ where $n = |\mathcal{X}|$. Hence, given the set of balls and the clustering induced by the balls $(\mc C,\mc B)$, we can test in $O(n^2+k^2)$ time if the clustering satisfies the niceness conditions.
\end{proof}

%%%%%%%%%%%%%%%%%%%%%%%%%%%%%%%%%%%%%%%%%%%%%%%%%%%%%%%%%%%%%%%%%%%%%%%%%%%%%%%%%%%%%%%%
\section{Algorithm}
\subsection{Based on the knowledge of $\max r(B_i)$}
Given a set $\mc X$, we assume that our algorithm knows the value of the maximum radius. Iteratively, we find the densest ball of twice the maximum radius. Delete some points and repeat till we find $k$ such balls. The algorithm is described in detail below.

\begin{algorithm}
\begin{alg}
%\item[] for $i=1$ to $l$
%\begin{itemize}
%	\item[] Let $L_i$ denote the set of balls found in $i^{th}$ iteration. Initially $L_i=\phi$
%	\item[] set $d=d_i$
	\item[] Set $r = \max_i \thinspace r(B_i)$
	\item[] for $i=1$ to $k$
	\begin{itemize}
		\item[] Let $B_i'$ be the densest ball of radius $2r$ such that $c(B_i') \in \mathcal{X}$.
		\item[] Set $\mc X=\mc X\setminus B_{4r}(c(B_i'))$. 
	\end{itemize}
	%\item[] If the clustering $L_i$ is nice
	%\begin{itemize}
%	\item[] $L := L \cup L_i$.
	%\end{itemize}
%\end{itemize}
\item[] Let $\mc B' := \{B_1',\ldots,B_k'\}$ and $c(\mc B') = \{c(B_1'),\ldots,c(B_k')\}$.
\item[] Output $\mc B'$ and the clustering $\mc C'$ induced by $c(\mc B')$.
\label{alg:NotKnown}
\end{alg}
\caption{Alg. for known $\max{r(B_i)}$}
\end{algorithm}

\begin{theorem}
Given a $(2,8)$-nice set $\mc X$. Let $\mc B = \{B_1,\ldots,B_k\}$ be the set of any balls that satisfy the separation and sparseness condition. Given $r = \max r(B_i)$, then Algorithm \ref{alg:NotKnown} outputs a set $\mc B' = \{B_1',\ldots,B_k'\}$ such that 
\begin{itemize}
	\item The sets $\mc B$ and $\mc B'$ are unique under intersection. That is there exists a permutation of the set $\mc B$ such that, for all $i\neq j$, $B_i \cap B_j' = \phi$ and for all $i$, $B_i \cap B_i' \neq \phi$
	\item The clustering $\mc C'$ induced by $\mc B'$ is such that $\mc C'|_\mc B = \{B_1,\ldots,B_k\}$.	
\end{itemize}

\end{theorem}
\begin{proof} Let $B_1',\ldots,B_i'$ be the balls found till the $i^{th}$ iteration. Also, observe that after every iteration we delete some points from $\mc X$. Let $\mc X_i$ denote the points in $\mc X$ after the $i^{th}$ iteration. 

\textit{Induction hypothesis}: For all $1\le i \le l$, $B_i \cap B_i' \neq \phi$ and $X_l \cap \{\cup_{i=1}^l B_i\} = \phi$.

\textit{Base case}: We wish to prove that $B_1 \cap B_1' \neq \phi$. Assume for the sake of contradiction, that $B_1'$ is such that for all $i$, $B_1' \cap B_i = \phi$. Now, $B_1' \cap \{X \backslash \cup_{i\in[k]} B_i\} = B_1'$. Observe that $B_1'$ is the densest ball of radius $2r$ with $c(B_1') \in \mathcal{X}$. Hence, for all $i$, $|B_1'| \ge |B_i|$. Hence, $|B_1'| \ge \min_{i=1}^k |B_i|$. Thus, $|B_1' \cap \{X \backslash \cup_{i\in[k]} B_i\}| \ge \min |B_i|$ which contradicts the fact that $\mc B$ satisfies the sparseness condition. Thus, $\exists i$ s.t. $B_i \cap B_1' \neq \phi$. Without loss of generality, let $B_i = B_1$. Now, $r(B_1) \le r$, $r(B_1') = 2r$ and $B_1 \cap B_1' \neq \phi$, hence $|c(B_1')-c(B_1)| \le 3r$. Thus, $B_1 \subseteq B_{4r}(c(B_1'))$ and hence $X_1 \cap B_1 = \phi$. 

\textit{Induction step}: We are given that for all $1\le i \le l$, $B_i \cap B_i' \neq \phi$. We wish to prove that $B_{l+1} \cap B_{l+1}' \neq \phi$. The proof for the induction step follows the exact same arguments as the base case.

Hence, we get that for all $i$, $B_i \cap B_i' = \phi$. Separation $\nu = 8$ implies that for all $j\neq i$, $B_i' \cap B_j' = \phi$ and $B_i' \cap B_j = \phi$. This completes the proof of the first part of the theorem.

Let $x \in B_i$. Observe that, $|c(B_i)-c(B_i')| \le 3r$ and for all $j\neq i$, $|c(B_i')-c(B_j)| > 5r$. Now, $|x-c(B_i')| \le |x-c(B_i)| + |c(B_i')-c(B_i)| \le r+3r = 4r$. For $j \neq i$, $|x-c(B_j')| \ge |c(B_j')-c(B_i)| - |x-c(B_i)| > 5r - r = 4r$. Hence, we see that the clustering $\mc C'$ assigns $x$ to the $i^{th}$ cluster. Hence, $\mc C'|_\mc B = \{B_1,\ldots,B_k\}$.
\end{proof}

\begin{theorem}
Algorithm \ref{alg:NotKnown} runs in time $O($poly$(|\mathcal{X}|,k))$.
\end{theorem}
\begin{proof}
During each iteration, we find the densest ball of radius $2r$. For each point in $\mathcal{X}$, we count the number of points whose distance is $\le 2r$. This can take $O(n)$ in the worst case. Hence, finding the densest ball takes $O(n^2)$ time. Similarly, deleting points from $\mathcal{X}$ takes $O(n^2)$ time. Hence, the for-loop takes $O(n^2k)$ time.
\end{proof}

\subsection{Based on the knowledge of $\min |B_i|$}
%All algoritthms are polynomial in n and k
In this case, we assume that the algorithm doesn't know the value of $r = \max r(B_i)$. We assume that the algorithm knows the number of points in the smallest cluster $\min |B_i| = t$ (say). This algorithm works in two phases.

\begin{algorithm}
\begin{alg}
\item[] \textbf{Phase 1:}
%\begin{itemize}
\item[] Let $\mc S$ denote the balls found so far. Initialize $\mc S = \phi$.
\item[] Let $D$ be the sorted list of distances between all pairs of points in $\mc X$. $D = \{d_1,\ldots,d_t\}$. Note that $t \le |\mc X|^2/2$.
\item[] for $i=1$ to $t$
\begin{itemize}
\item[] foreach $x \in \mc X$
\begin{itemize}
\item[] Let $B$ be a ball of radius $d_i$ centered at $x$.
\item[] if $|B| \ge t$ and $|B \cap \mathcal{S} | = \phi$
\begin{itemize}
\item[] $\mathcal{S} = \mathcal{S} \cup B$. 
\item[] $\mc X = \mc X\setminus B$.
\end{itemize}
\end{itemize}
\end{itemize}
\item[] Output $\mc S := \{B_1',\ldots,B_p'\}$ to the second phase.
%\end{itemize}

\item[] \textbf{Phase 2:}
%\begin{itemize}
\item[] Let $L_C$ denote the list of centers of balls found in the first phase. That is, $L_C = (c_1,\ldots,c_p)$ where $c_i = c(B_i')$.
\item[] Let $\mc L$ denote the list of clusterings found so far. Initialize $\mc L := \phi$
\item[] Let $\mc C_i$ denote the current clustering. Initialize $\mc C_k = \{c_1,\ldots,c_k\}$.
\item[] for $i=k+1$ to $p$
\begin{itemize}
\item[] Let $\mc D$ denote the $(k+1)$-clustering given by $\mc D = \mc C_{i-1} \cup c_i$. 
\item[] Let $c_r, c_s$ ($r < s$) denote the closest two clusters in $\mc D$. That is $$(c_r, c_s) = \argmin_{c_u,c_v \in \mc D, u < v} |c_u -c_v|$$
\item[] $\mc C_i = \mc D \setminus c_s$.
\item[] $\mc L = \mc L \cup \mc C_i$.
%\end{itemize}

%\item[] // We built a forest with $k$ trees corresponding to the $k$ different clusters we found. 
%\item[] Use single linkage to merge the $k$-clusters found so far. Update $T$.
\end{itemize}
\item[] Output $\mc L$ %and $T$.
\label{alg:Known}
\end{alg}
\caption{Alg. for known $\min{B_i}$}
\end{algorithm}

Before we prove any results about the algorithm, we need to define some quantities and conventions that will be used in the proof.
\begin{itemize}
\item $\mc B = \{B_1, \ldots, B_k\}$ - The set of balls in the dataset $\mc X$ that satisfy the separation and sparseness condition (Defn. \ref{defn:niceness})
\item $\mc S = \{B_1', \ldots, B_p'\}$ - The set of balls found in the first phase of Alg. \ref{alg:Known}. 
\item $L_C = \{c_1, \ldots, c_p\}$ - The set of centers of the balls in $\mc S$.  
\end{itemize}
Note that we denote the balls that were found by the Alg. \ref{alg:Known} as $B_i'$ and their centers by $c_i$. 
\begin{itemize}
\item $\mc L = \{\mc C_k, \ldots, \mc C_p\}$ - The set of possible clusterings outputed by Alg. \ref{alg:Known}. 
\end{itemize}

\begin{definition}[Good Index]
\label{defn:goodIdx}
Let $\mc B, \mc L$ and $\mc S$ be as defined above. We say that an index $g$ is good if it has the following properties.
\begin{itemize}
\item $\forall i \le g$, $\exists m$ such that $B_i' \cap B_m \neq \phi$
%\item $\forall m$, $B_{g+1}' \cap B_m = \phi$
\item $\forall m$, $\exists i \le g$ such that $B_i' \cap B_m \neq \phi$
\end{itemize}
The next lemma shows that there exists a good index.
\end{definition}

\begin{lemma}
Given a $(2,12)$-nice set $\mc X$. Let $\mc B, \mc L$ and $\mc S$ be as defined above. Define $r:= \max r(B_i)$. Then there exists an index $g$ which is good (Defn. \ref{defn:goodIdx}).
\label{lemma:goodIdx}
\end{lemma}

\begin{proof}
Let $\mc T = \{i: B_i' \cap B_m = \phi$ for all $m\}$. If $\mc T = \phi$ then let $q = p+1$. Otherwise let $q = \min(\mc T)$. We will claim that $g = q-1$ is the required good index in both cases.

From the definition of the set $\mc T$ and $g$, we know that $\forall i \le g$, $B_i' \cap B_m \neq \phi$ for some $m$. Hence, we see that the index $g$ satisfies the first goodness criteria. We prove that it also satisfies the second criteria by contradiction. Assume that there exists $l$ such that $\forall i \le g$, $B_i' \cap B_l = \phi$.

Let $B$ be some ball with center $x \in B_l$ and radius $r_B \le \max_{y \in B_l} |x-y| \le 2r(B_l)$. Then $B$ contains atleast $t$ points. Observe that since $B_t' \cap B_m = \phi$ for all $m$ and $\mc X$ is $(2,12)$-nice, $r(B_t') > 2r$. Hence, Alg \ref{alg:Known} adds $B$ to $\mc S$ before it adds $B_t'$. Hence, $B = B_i'$ for some $i\le g$. This contradicts the assumption that $B_i' \cap B_l = \phi$ for all $i \le g$. Hence, we get that there exists a good index. 
\end{proof}


\begin{lemma}
Given a $(2,12)$-nice set $\mc X$. Let $\mc B, \mc S$ and $\mc L$ be as defined above. Given $t = \min |B_i|$. Denote by $r := \max r(B_i)$. Let $g = \min \{i: i$ is a good index\}. Then for all $i\le g$, $r(B_i') \le 2r$.

\label{lemma:centerDist}
\end{lemma}

\begin{proof}
Let $\mc T = \{B_1', \ldots, B_g'\}$. Alg. \ref{alg:Known} considers the radius of balls in sorted order. Hence, $r(B_1') \le r(B_2') \le \ldots \le r(B_g')$. Since $g$ is good, there exists $B_l$ such that $B_g' \cap B_l \neq \phi$. The claim is that $B_g'$ is the only ball in $\mc T$ which intersects $B_l$. Assume that this is not the case. Then there exists $h < g$ such that $B_h' \cap B_l \neq \phi$. Hence, the index $g-1$ is good. This contradicts the minimality of $g$. 

Hence, $B_g'$ is the only ball that intersects $B_l$. Now, we claim that $\forall i \le g, r(B_i') \le 2r(B_l)$. Assume that this is not the case. Let $q$ be the first index such that $r(B_q') > 2r(B_l) \ge r(B_{q-1})$. Let $B$ be some ball with center $x \in B_l$ and radius $r_B \le \max_{y \in B_l} |x-y| \le 2r(B_l)$. Then $B$ contains atleast $t$ points. Hence, Alg \ref{alg:Known} adds $B$ to $\mc S$ before it adds $B_q'$. Hence, $B$ is equal to one of $B_1', \ldots, B_{q-1}'$. This contradicts the fact that $\forall i \le g-1$, $B_i' \cap B_l = \phi$. Thus, $r(B_g') \le 2r(B_l) \le 2r$. 
\end{proof}

\begin{corollary}
Given a $(2,12)$-nice set $\mc X$. Let $\mc B, \mc S, L_C$ and $\mc L$ be as defined above. Given $t = \min |B_i|$. Denote by $r := \max r(B_i)$. Let $g = \min \{i: i$ is a good index\}. Then for all $i,j \le g$
\begin{itemize}
\item If $B_i' \cap B_m \neq \phi$ and $B_j' \cap B_m\neq\phi$ for some $m$, then $|c_i-c_j| \le 6 r(B_m) \le 6r$.
\item If $B_i' \cap B_m \neq \phi$ and $B_j' \cap B_n\neq\phi$ for some $m,n$, then $|c_i-c_j| > 6r$.
\end{itemize}
\label{cor:centerDist}
\end{corollary}

\begin{proof}
We will use Lemma \ref{lemma:centerDist} and the  triangle equality to prove the two results. For the first case, $|c_i-c_j| \le |c_i-c(B_m)| + |c(B_m)-c_j| \le r(B_i') + r(B_m) + r(B_m) + r(B_j') \le 2r + r + r + 2r = 6r$. For the second case, $|c_i-c_j| \ge  |c(B_m)-c_j| - |c_i-c(B_m)| \ge |c(B_m)-c(B_n)| - |c(B_n) - c_j| - |c_i-c(B_m)| > 12r - 3r -3r = 6r$.
\end{proof}

\begin{theorem}
Given a $(2,12)$-nice set $\mc X$. Let $\mc B = \{B_1,\ldots,B_k\}$ be the set of any balls that satisfy the separation and sparseness condition. Given $t = \min |B_i|$, Algorithm \ref{alg:Known} outputs a list $\mc L$ of possible clusterings of size $|\mc L| \in O(|\mc X|)$. $\mc L$ contains a clustering $\mc C$ such that $\mc C|_{\mc B} = \{B_1,\ldots,B_k\}$.
\end{theorem}

\begin{proof}
$\mc L$ contains a list of clusterings $\mc C_k,\ldots,\mc C_p$. Let $g$ be the first index which has the properties as stated in Lemma \ref{lemma:goodIdx}. Let $\mc C_g$ be the corresponding clustering. We will show that the clustering $\mc C_g$ is such that $\mc C_g|_{\mc B} = \{B_1,\ldots,B_k\}$. 

Let $\mc C_g = \{c_{g(1)},\ldots,c_{g(k)}\}$ and let $B_{g(1)}',\ldots,B_{g(k)}'$ be the corresponding balls. The claim is that for every $m \le k$ there exists $i \le k$ such that $B_m \cap B_{g(i)}' \neq \phi$. For the sake of contradiction, assume that this is not the case. 

Hence, there exists ball $B_n$ (say) such that for all $i$, $B_{g(i)}' \cap B_n = \phi$ for all $i$. From the definition of $g$, we know that for all $i$, $B_{g(i)}$ intersects some ball $B_m$. Hence, from the pigeonhole principle we get that there exists indices $m, i$ and $j$ such that $B_{g(i)} \cap B_m \neq \phi$ and $B_{g(j)} \cap B_m \neq \phi$. From Cor. \ref{cor:centerDist}, we get that $|c_{g(i)}-c_{g(j)}| \le 6r$.

Fromt the definition of $g$, there exists some index $n_1$ such that $B_{n_1}' \cap B_n \neq \phi$. Let $n_1, \ldots, n_r$ be all the indices such that $B_{n_i}' \cap B_n \neq \phi$. Observe that by the $g^{th}$ iteration of ALgorithm \ref{alg:Known}, all the centers $c_{n_1}, \ldots, c_{n_r}$ were deleted. Let $c_{n_r}$ be the last center out of these that was deleted. Then for some $c_i$, $|c_i - c_{n_r}|$ was the minimum distance between any two clusters at some step (say $q^{th}$ step) of the algorithm. Note that $B_i' \cap B_n \neq \phi$. Hence, using Cor. \ref{cor:centerDist} we get that $|c_i - c_{n_r}| > 6r$. 

Observe that centers $|c_{g(i)}-c_{g(j)}| \le 6r < |c_i - c_{n_r}|$. However, $c_{g(i)}$ and $c_{g(j)}$ were not picked at the $q^{th}$ step of the algorithm. This is a contradiction. Hence, we get that $\forall m$, $\exists i$ such that $B_m \cap B_{g(i)} \neq \phi$. Hence, we get that the sets $\mc B$ and $\mc B' = \{B'_{g(1)}, \ldots, B'_{g(k)}\}$ are unique under intersection.

Let $x \in B_i$. WLOG let $B_{g(i)}'$ intersect $B_i$. Now using Lemma \ref{lemma:centerDist} and the triangle inequality, 
\vspace{-3mm}
\begin{align*}
|x-c(B_{g(i)}')| \le |x-c(B_i)| + |c(B_{g(i)}')-c(B_i)| \le |x-c(B_i)| + r(B_{g(i)}')+ r(B_i) \le r+ 2r + r = 4r.
\end{align*} 
For $j \neq i$, using Lemma \ref{lemma:centerDist} again we get that
\vspace{-3mm}
\begin{align*} 
|x-c(B_{g(j)}')| &\ge |c(B_{g(j)}')-c(B_i)| - |x-c(B_i)| \ge |c(B_i)-c(B_j)| - |c(B_{g(j)}')-c(B_j)| - |x-c(B_i)|\\
 &> 12r - r -3r = 9r.
\end{align*} 
Hence, we see that the clustering $\mc C_g$ assigns $x$ to the $i^{th}$ cluster. Hence, $\mc C_g|_\mc B = \{B_1,\ldots,B_k\}$.
 
\end{proof}


































%%%%%%%%%%%%%%%%%%%%%%%%%%%%%%%%%%%%%%%%%%
\end{document}
