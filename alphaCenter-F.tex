\documentclass[11pt]{article}
\usepackage[paper]{nickstyle}
\usepackage{color}
\usepackage{hyperref}
\usepackage{amssymb, amsmath}
\usepackage{enumitem}

\newcommand{\mc}{\mathcal}
\setlength{\parindent}{24pt}
\renewcommand{\bar}[1]{\overline{#1}}

\newtheorem{fact}[theorem]{Fact}
\newtheorem{conj}[theorem]{Conjecture}
\newtheorem{problem}[theorem]{Problem}
\newcommand{\ConjName}[1]{\label{con:#1}}
\newcommand{\Conj}[1]{Conjecture~\ref{con:#1}}
\newcommand{\ProblemName}[1]{\label{prob:#1}}
\newcommand{\Problem}[1]{Problem~\ref{prob:#1}}
\renewcommand{\dot}{\bullet}
\newcommand{\Tr}{\operatorname{tr}}
\newcommand{\eps}{\epsilon}

\newcommand{\lmax}{\lambda_\mathrm{max}}
\newcommand{\lmin}{\lambda_\mathrm{min}}
\newcommand{\ufinal}{u_\mathrm{final}}
\newcommand{\lfinal}{l_\mathrm{final}}
\newcommand{\umax}{u_\mathrm{max}}

\newcommand{\Symraw}{\mathbb{S}}
\newcommand{\Sym}[1][]{\Symraw^{\ifthenelse{\equal{#1}{}}{m}{#1}}}
\newcommand{\Psd}[1][]{\Symraw_+^{\ifthenelse{\equal{#1}{}}{m}{#1}}}
\newcommand{\Reals}{\mathbb{R}}
\newcommand{\iprod}[2]{\langle #1, #2 \rangle}
\newcommand{\paren}[2][]{#1({#2}#1)}
\newcommand{\qform}[2]{\transp{#2}#1#2}
\newcommand{\transp}[1]{#1^T}



% Simple (outer) environment for algorithms
\newenvironment{outer_alg}{
    \begin{list}{}{
        \setlength{\itemsep}{2pt}
        \setlength{\parsep}{0pt}
        \setlength{\parskip}{0pt}
        \setlength{\topsep}{1pt}
        \setlength{\leftmargin}{5pt}
    }
}
{
    \end{list}
}

% Simple environments for algorithms
\newenvironment{alg}{
    \begin{list}{}{
        \setlength{\itemsep}{2pt}
        \setlength{\parsep}{0pt}
        \setlength{\parskip}{0pt}
        \setlength{\topsep}{1pt}
    }
}
{
    \end{list}
}

%%%%% Title %%%%%
\title{\LARGE Centre Proximity with sparse noise}
\author{}


%%%%% Document Body %%%%%
\begin{document}
\maketitle

\section{Problem Statement}
We are given a data set $\mc X$ with euclidean distance $d$. For any set $\mc A\subseteq \mc X$ we define the centre and radius of $\mc A$ respectively as follows:
$$
\begin{array}{lr}
c(\mc A)= \frac{\sum_{x\in \mc A} x}{\lvert \mc A\rvert }, & r(\mc A) = \max_{x \in \mc A} d(x, c_i)
\end{array}
$$
We denote a ball of radius $r$ at centre $x$ by $B_r(x)$.

% Adding C_{k+1} as garbage collector?
\begin{definition}
\label{defn:alphacpnoise}
Set $\mc X$ is defined to be $(\alpha, \eta)$-center nice if there exist $\mc S \subseteq \mc X$ such that 
let $\mc C= \{ C_1, \ldots, C_k \}$ be a clustering of $\mc S$ induced by centers $c_1,\ldots,c_k$. We say that clustering $\mc C$ has $(\alpha, \eta)$-center proximity w.r.t $\mc X, \mc S$ and $k$ if the following holds. 
\begin{itemize}[nolistsep, noitemsep]

\item[$\diamond$] {\bf $\alpha$-centre proximity}: For all $x \in C_i$ and $i\neq j$, $\thinspace\alpha d(x, c_i) < d(x, c_j)$
\item[$\diamond$]{\bf $\eta$-sparse noise}: For any ball $B$ such that $c(B)\in \mathcal{X}$, if $r(B)\leq \eta\max \thinspace r(C_i)$, then $|B\cap (\mc X\setminus \mc S)| < \min_i |C_i|/2$
\end{itemize}
\end{definition}

\subsection{Goal}
Given a data set $\mc X$ and parameters $k$ and $t$, our goal is to find a $k$-clustering $C^*=\{C^*_1,\ldots,C^*_k\}$ of $\mc X$ such that $\min_{i} \lvert C^*_i \rvert  \geq t$.
%%%%%%%%%%%%%%%%%%%%%%%%%%%%%%%%%%%%%%%%%%

\section{Algorithm}
Balcan and Liang studied the problem of clustering in the presence of background noise. They presented a polynomial-time algorithm that clusters an input data which has $\epsilon$ fraction of it as noisy points. In this work, we consider a similar problem but with a different assumption on the noise. We do not place any restriction on the size of the noise but assume that it doesn't have any structure. 

Similar to Blacan and Liang setting, our algorithm takes as input the data set $\mc X$, the number of clusters $k$ and parameter $t$ defined as the number of points in the cluster with minumum size. Let $\mc C'$ be a given clustering of the set $\mc X$ and $p, q \in \mc X$. Let $B_{p,q} = B(p, d(p, q))$. Define $Y_{p,q} := \{C' \in \mc C' : C' \subseteq B_{p,q} \text{ or } |B_{p,q} \cap C'| \ge t/2\}$. 
%For any $x \in \mc X$, let $B_x$ denote the ball centered at $x$ of radius atmost $r$. Define $\mc D_{p,q} = \{x \in \mc X\setminus B : |B_x| \ge t/2 \text{ and } \exists p' \in B \text{ such that } (\alpha-1) d(p', p) \ge d(p', x)\}$. 
We say that the ball $B_{p,q}$ satisfies the \textbf{sparse distance condition} w.r.t clustering $\mc C'$ when the following holds.
\begin{itemize}[noitemsep, leftmargin=*]
\item $|B_{p,q}| \ge t$.
\item For any $C' \in \mc C'$, %if $C' \cap B_{p,q} \neq \phi$, then either $C' \subseteq B_{p,q}$ or $|B_{p,q} \cap C'| \ge t/2$. In other words, 
if $C' \cap B_{p,q} \neq \phi$, then $C' \in Y_{p,q}$.
%\item $|D_{p,q}| \le \epsilon|\mc X|$
	%\begin{itemize}[nolistsep, noitemsep]
	%	\renewcommand\labelitemii{$\diamond$}
	%	\item for all $p' \in B$, $(\alpha-1) d(p', p) < d(p', x)$
	%	\item $|B_x| < t/2$
	%	\item
	%\end{itemize} 
\end{itemize}

Intuitively, the algorithm (Alg. \ref{alg:alphacp}) works as follows. It starts with each point in its own cluster. It then iterates over all pairs of points $p, q$ in increasing order of their distance $d(p, q)$. If $B(p, d(p,q))$ satisfies the sparse distance condition, then it merges all the clusters which intersect with the ball into a single cluster. We will show that for all $\mc S \subseteq \mc X$ and for all clusterings $\mc C$ of $\mc S$ which have $(\alpha, \eta)$-center proximity and $\min |C_i| = t$, Alg. \ref{alg:alphacp} outputs a tree which respects the clustering $\mc C$.

\begin{algorithm}[!ht]
\begin{alg}
	\item \textbf{Input: } $(\mc X, d), k$ and $t$
	\item \textbf{Output: } A hierarchical clustering tree $T'$ of $\mc X$.
	\item[1] Let $\mc C'$ denote the current clustering of $\mc X$. Initialize $\mc C'$ so that all points are in their own cluster. That is, $\mc C' = \{ \{x\}: x \in \mc X\}$.
	\item[2] Iterate over all pairs of points $p, q$ in increasing order of the distance $d(p, q)$. If $B(p, d(p, q))$ satisfies the sparse distance condition then
	\begin{itemize}
	\renewcommand\labelitemi{}
		\item $C_{temp} = \phi$. 
		\item for all $C' \in Y_{p,q}$
		\begin{itemize}
		\renewcommand\labelitemii{}
			\item $\mc C' = \mc C' \setminus C'$ and $C_{temp} = C_{temp} \cup C'$
		\end{itemize}
		\item $\mc C' = \mc C' \cup C_{temp}$.
		\item This step basically merges all the clusters in $Y_{p, q}$ into a single cluster.
	\end{itemize}
	\item[3] Output clustering tree $T'$. The leaves of $T'$ are the points in dataset $\mc X$. The internal nodes corresppond to the merging performed in the previous step.
	%\item[4] Construct $T$ from $T'$ by deleting the all nodes which do not have any children.
\end{alg}
\caption{Alg. for $(\alpha, \eta)$-center proximity with parameter $t = \min_i |C_i|$}
\label{alg:alphacp}
\end{algorithm}


\begin{theorem}
Given a clustering instance $(\mc X, d)$, the number of clusters $k$ and a parameter $t$. Alg. \ref{alg:alphacp} outputs a tree $T'$ with the following property. For all $\mc S \subseteq \mc X$ and for all clusterings $\mc C = \{C_1, \ldots, C_k\}$ of $\mc S$ induced by centers $c_1, \ldots, c_k \in \mc X$ which satisfy $(\alpha, \eta)$-center proximity w.r.t $\mc X, \mc S$ and $k$. 

If $\alpha \ge 2 + \sqrt 7$, $\eta \ge 1$ and $ \min_i|C_i| = t \ge 2$ then for every $1\le i \le k$, there exists a node $N_i$ in the tree $T'$ such that $C_i \subseteq N_i$ and for $j \neq i$, $C_j \cap N_i = \phi$ . That is, $N_i$ contains points from only one of the good clusters. 
\end{theorem}

\begin{proof}
Fix any $\mc S \subseteq \mc X$. Let $\mc C = \{C_1, \ldots, C_k\}$ be a clustering of $\mc S$ such that $\min |C_i| = t$ and $\mc C$ has $(\alpha, \eta)$-center proximity. Denote by $r_i := r(C_i)$ and define $r := \max r_i$. Let $\mc C' = \{C_1', \ldots, C_{k'}'\}$ be the current clustering of $\mc X$ (as defined in Alg. \ref{alg:alphacp}). Let $p, q \in \mc X$ and denote by $B_{p, q} = B(p, d(p, q))$. Note that whenever $B_{p, q}$ satisfies the sparse-distance condition, all the clusters in $Y_{p, q}$ are merged together and the clustering $\mc C'$ is updated. Throughout the proof, we will denote by $C_i \in \mc C$ the clusters of the set $\mc S$ (also called `good' clusters) and by $C_i' \in \mc C'$ the clusters of the set $\mc X$.

We will prove the theorem by proving two key facts. First, we will show that at any step, if the algorithm merges points from a good cluster $C_i$ with points from another good cluster $C_j$, then at that step the distance being considered $d = d(p,q) > r_i$. Then, we show that when the algorithm considers the distance $d = d(c_i, q_i) = r_i$, it merges all points from the good cluster $C_i$ (and possibly points from $\mc X\setminus \mc S$) into a single cluster $C_i'$. Hence, there exists a node in the tree $N_i$ which contains all the points from $C_i$ and no points from any other good cluster $C_j$. 

\noindent\textit{\underline{Proof of Fact 1}}\\
Consider the first merge step which merges points from a good cluster $C_i$ with points from some other good cluster. Let $p, q \in \mc X$ be the pair of points being considered at this step and $B_{p,q} = B(p, d(p, q))$ the ball that satisfies the sparse distance condition at this merge step. Then, there exists a cluster $C_i' \in Y_{p,q}$ such that $C_i \cap C_i' \neq \phi$ and for all $n \neq i$, $C_n \cap C_i' = \phi$. Also, there exists cluster $C_j' \in Y_{p, q}$ such that $C_j' \cap C_j \neq \phi$ for some $C_j$ and $C_j' \cap C_i = \phi$.

We need to show that $d(p, q) > r_i$. However, before we can prove that we need another result which is proved as a claim below. We then prove the desired result (Claim \ref{claim:maxrirj}).
\begin{claim}
\label{claim:fromBothCluster}
Let $p, q \in \mc X$, $B_{p, q}$, $C_i, C_i', C_j'$ be as defined above. If $d(p, q) \le r,$ then $B_{p, q} \cap C_i \neq \phi$ and there exists $n \neq i$ such that $B_{p, q} \cap C_n \neq \phi$.
\end{claim}
\vspace{-0.1in} $C_i' \in Y_{p, q}$. In the first case, if $C_i' \subseteq B_{p, q}$ then $B_{p,q}$ contains points from $C_i$ by set inclusion. In the second case assume that $|C_i' \cap B_{p,q}| \ge t/2$. For the sake of contradiction, assume that $B_{p, q}$ contains no points from $C_i$. That is, $B \cap C_i' \subseteq C_i' \setminus C_i \subseteq \mc X \setminus \mc S$. This implies that $B \cap C_i' \subseteq B \cap \{\mc X \setminus \mc S\}$. Hence, $|B\cap \{\mc X \setminus \mc S\}| \ge |B \cap C_i'| > t/2$. This contradicts the sparse noise assumption in Defn. \ref{defn:alphacpnoise}. Hence, $B_{p, q} \cap C_i \neq \phi$.

$C_j' \in Y_{p, q}$. In the first case, if $C_j' \subseteq B_{p, q}$ then $B_{p,q}$ contains points from some $C_j$ by set inclusion. In the second case assume that $|C_j' \cap B_{p,q}| \ge t/2$. For the sake of contradiction, assume that for all $n \neq i$, $B_{p, q}$ contains no points from $C_n$. That is, $B_{p, q} \cap C_j' \subseteq C_j' \setminus (\cup_{n \neq i} C_n) \subseteq \mc X \setminus \mc S$. This implies that $B_{p, q} \cap C_j' \subseteq B_{p,q} \cap \{\mc X \setminus \mc S\}$. Hence, $|B_{p, q}\cap \{\mc X \setminus \mc S\}| \ge |B_{p,q} \cap C_i'| > t/2$. This contradicts the sparse noise assumption in Defn. \ref{defn:alphacpnoise}. Hence, there exists $C_n \neq C_i$ such that $B_{p, q} \cap C_n \neq \phi$.

\begin{claim}
\label{claim:maxrirj}
Let the framework be as given in Claim \ref{claim:fromBothCluster}. Then, $d(p, q) > r_i$.
\end{claim}

\vspace{-0.1in} If $d(p, q) > r$, then the claim follows trivially. We assume that $d(p, q) \le r$. From claim \ref{claim:fromBothCluster}, $B_{p, q}$ contains $p_i \in C_i$ and $p_j \in C_j$. Let $r_i = d(c_i, q_i)$ for some $q_i \in C_i$.
\begin{align*}
d(c_i, q_i) &< \frac{1}{\alpha} d(q_i, c_j) \le \frac{1}{\alpha} \bigg[ d(p_j, c_j) + d(p_i, p_j) + d(p_i, q_i)\bigg] < \frac{1}{\alpha} \bigg[ \frac{1}{\alpha}d(p_j, c_i) + d(p_i, p_j) + 2d(c_i, q_i)\bigg]\\
& < \frac{1}{\alpha} \bigg[ \frac{1}{\alpha}d(p_i, p_j) + \frac{1}{\alpha}d(c_i, q_i) + d(p_i, p_j) + 2d(c_i, q_i)\bigg]
\end{align*}
This implies that $(\alpha^2 - 2\alpha - 1)d(q_i, c_i) < (\alpha + 1) d(p_i, p_j)$. For $\alpha \ge 2 + \sqrt 7$, this implies that $d(c_i, q_i) < d(p_i, p_j)/2$. Now, using triangle inequality, we get that $d(c_i, q_i) < d(p_i, p_j)/2 \le \frac{1}{2}[d(p, p_i) + d(p, p_j)] < d(p, q)$.\\

\noindent\textit{\underline{Proof of Fact 2}}\\
Consider the merge step when $p = c_i$ and $q = q_i$ such that $d(p, q) = r_i$. We will prove that the ball $B(c_i, q_i)$ satisfies the sparse-distance condition. Hence, all the points from the good cluster $C_i$ will be merged together. Note that this step merges all the clusters in $Y_{c_i, q_i}$. Hence to complete the proof, we also show that all the clusters in $Y_{c_i, q_i}$ don't intersect with any other good cluster $C_j$. We state this formally below and then prove it.

\begin{claim}
%\vspace{-0.1in}
\label{claim:dciqi}
Let $c_i \in \mc X$ denote the center of the cluster $C_i$ and $q_i$ is such that $d(c_i, q_i) = r_i$. Then, $B(c_i, r_i)$ satisfies the sparse distance condition and for all $C' \in Y_{c_i, q_i}$, for all $j \neq i, C' \cap C_j = \phi$.
\end{claim}

\vspace{-0.1in} Denote by $B = B(c_i, q_i)$. $|B| = |C_i| \ge t$. Observe that, for all $C' \in \mc C'$, $|C'| = 1$ or $|C'| \ge t$. We need to prove two statements. If $C' \cap B \neq \phi$, then $C' \in Y_{p,q}$ and $C' \cap C_j = \phi$. 

\begin{itemize}[nolistsep]
\item Case 1. $|C'| = 1$. If $C' \cap B \neq \phi \implies C' \subseteq B$. Hence, $C' \in Y_{p,q}$ and for all $j \neq i$, $C' \cap C_j = \phi$

\item Case 2. $|C'|\ge t$. $C' \cap B \neq \phi$. Let $h(C')$ denote the height of the cluster in the tree $T'$. Now, we consider two subcases.
\begin{itemize}
\renewcommand\labelitemii{$\circ$}
\item Case 2.1. $h(C') = 1$. In this case, there exists a ball $B'$ such that $B' = C'$. We know that $r(B') \le r_i \le r$. Hence using Claim \ref{claim:maxrirj}, we get that for all $j \neq i$, $B' \cap C_j = \phi$. Thus, $|B'\setminus C_i| \le t/2 \implies |B\cap C'| = |C'| - |C'\setminus B| = |C'| - |B'\setminus C_i| \ge t/2$. Hence, $C' \in Y_{c_i, q_i}$.

\item Case 2.2. $h(C') > 1$. Then there exists some $C''$ such that $h(C'') = 1$ and $C'' \subset C'$. Now, using set inclusion and the result from the first case, we get that $|B\cap C'| \ge |B\cap C''| \ge t/2$. Hence, $C' \in Y_{c_i, q_i}$. Using Claim \ref{claim:maxrirj}, we get that for all $j \neq i$, $C' \cap C_j = \phi$.
\end{itemize} 
\end{itemize}
%Let $x \in D_{c_i, q_i}$. We claim that $x \not\in \mc S$. For the sake of contradiction, let us assume that $x \in C_j$ for some $j \neq i$. Now, $C_i$ and $C_j$ satisfy $\alpha$-center proximity. Using, Fact \ref{fact:alphacpDist}, we have that $\forall p' \in B$, $(\alpha - 1)d(p', c_i) < d(p', x)$. This contradicts the fact that $x \in D_{c_i, q_i}$. Hence, we get that $x \not\in \mc S$ which implies that $x \in \mc T$ (Defn. \ref{defn:alphacpnoise}). Thus, $D_{c_i, q_i} \subseteq \mc T$ and $|D_{c_i, q_i}| \le |\mc T| \le \epsilon |\mc X|$.  
\end{proof}

\begin{theorem}
Given clustering instance $(\mc X, d)$, $k$ and $t$. Algorithm \ref{alg:alphacp} runs in $O(|\mc X|^3)$time.
\end{theorem}

\begin{proof}
Let $n = |\mc X|$. Let $\mc C' =\{C_1', \ldots, C_{k'}'\}$ denote the current clustering of $\mc X$ as defined in Alg. \ref{alg:alphacp}. Observe that given the ball $B_{p, q} = B(p, d(p, q))$, for any cluster $C' \in \mc C'$, checking if $C' \in Y_{p, q}$ takes time O($|C'|$). Doing this for all clusters takes O(n) time. The algorithm examines all the pairs of points and hence runs in $O(n^3)$ time.
\end{proof}
%\begin{fact}
%\label{fact:alphacpDist}
%Let $\mc C = \{C_1, \ldots, C_k\}$ satisfy $\alpha$-center proximity. Then, for all $p \in C_i$ and $p'\in C_j$, $(\alpha-1) d(p, c_i) < d(p, p')$.
%\end{fact}
%\begin{proof}
%\vspace{-0.1in}\noindent The proof appeared in Awasthi, Blum and Sheffet [?].  
%\end{proof}


%%%%%%%%%%%%%%%%%%%%%%%%%%%%%%%%%%%%%%%%%%%%%%%%%%%%%%%%%%%%%%%%%%%

\end{document}

