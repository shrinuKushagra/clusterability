\documentclass[anon,12pt]{colt2016} % Anonymized submission
\def\COMPLETE{}
\usepackage{algorithm}
\usepackage{algorithmic}

\usepackage{nickstyle_ICML}

\usepackage{hyperref}
\usepackage[toc,page]{appendix}

\usepackage{color}
\usepackage[toc,page]{appendix}
\usepackage{xspace}
\usepackage{enumitem}
\usepackage{times}

\usepackage{float}
\usepackage{caption2}

\usepackage{tikz}
\usetikzlibrary{shapes, calc, arrows, through, intersections, decorations.pathreplacing, patterns}

\newtheorem{fact}[theorem]{Fact}
\newtheorem{conj}[theorem]{Conjecture}
\newtheorem{problem}[theorem]{Problem}
\newtheorem{claim}[theorem]{Claim}

% Simple environments for algorithms
\newenvironment{alg}{
    \begin{list}{}{
        \setlength{\itemsep}{2pt}
        \setlength{\parsep}{0pt}
        \setlength{\parskip}{0pt}
        \setlength{\topsep}{1pt}
    }
}
{
    \end{list}
}

\floatname{algorithm}{Algorithm}
\makeatletter
\renewcommand{\floatc@ruled}[2]{\vspace{3pt}{\@fs@cfont #1:\:} #2 \par \vspace{2pt}}
\makeatother


\newcommand{\mc}{\mathcal}
\newcommand{\mb}{\mathbf}

\newcommand{\todo}{\textcolor{blue}{[TODO]}\xspace}
\newcommand{\complete}{\textcolor{red}{[TO BE COMPLETED]}\xspace}
\newcommand{\wip}{\textcolor{red}{[Work in progress]}\xspace}
\newcommand{\samira}{\textcolor{blue}{[Samira]}\xspace}
\newcommand{\shrinu}{\textcolor{blue}{[Shrinu]}\xspace}

\newcommand{\fix}{\textcolor{blue}{[FIX]}\xspace}
\newcommand{\q}{\textcolor{blue}{[?]}\xspace}

 \coltauthor{\Name{Shai Ben-David} \Email{shai@cs.uwaterloo.ca }\\
 \addr University of Waterloo
 \AND
 \Name{Shrinu Kushagra} \Email{skushagr@uwaterloo.ca }\\
 \addr University of Waterloo
 \AND
 \Name{Samira Samadi} \Email{ssamadi6@gatech.edu}\\
 \addr Georgia Institute of Technology
 }


\title[Sparse Noise]{Clustering under unstructured noise}
\begin{document}

\maketitle

\begin{abstract}

\end{abstract}

\begin{keywords}
Clustering, $\alpha$-center proximity, $\lambda$-center separation
\end{keywords}

\section{Introduction}

\subsection{Motivation}
\todo
\subsection{Related Work}

In the past few years, there has been a line of work on defining notions of clusterability. The goal of all these methods has been to show that clustering will be computationally efficient for inputs that enjoy some nice structure.

.\cite{bilu2012stable} consider an input of an optimization problem to be \emph{stable} if small perturbations of the input do not change the optimal solution. Specially, they give an efficient algorithm to find the max-cut of sufficiently stable graphs.

In the same direction, \cite{awasthi2012center, balcan2012clustering}, consider a similar condition on the input of center-based clustering tasks. They propose efficient algorithms that output the optimal clustering of the data. \cite{ackerman2009clusterability} consider an imput such that the cost of the optimal clustering does not change under small additive perturbations of the input. They design an efficient algorithm that outputs a clustering with near-optimal cost. 

All the clusterability notions discussed so far assume that the  entire data set enjoys some nice property, i.e., made up of groups of similar objects. A more realistic setting would be to the following. The input dataset has some nice structure. However, this structure is disturbed by the addition of unstructured points (or {\it noise}). The {\it noise} makes it harder to detect the 

\shrinu\wip

\subsection{Structure of paper}
\shrinu\todo Following is the proposed structure of the paper.
\vspace{-0.1in}\begin{enumerate}
\setlength\itemsep{0em}
\item Introduction
\item Notations and definition
\item $\alpha$-center proximity	\vspace{-0.1in}
	\begin{enumerate}[leftmargin=32pt]
	\renewcommand{\labelenumii}{\theenumii}
	\renewcommand{\theenumii}{\theenumi.\arabic{enumii}.}
	\setlength\itemsep{0em}
	\item[\ref{section:alphaLowerBoundArbitrary}] Assuming arbitrary noise: lower Bound $\alpha \le 2\sqrt{2} + 3$ $(\approx 5.8)$
	\item[\ref{section:positiveResultSparseNoise}] Assuming sparse noise: Positive result for $\alpha \ge 2+\sqrt{7}$ $(\approx 4.6)$
	\item[\ref{section:alphaLowerBoundSparse}] Assuming sparse noise: lower bound $\alpha \le 2+\sqrt{3}$ $(\approx 3.7)$
	\end{enumerate}
\item $\lambda$-center separation \vspace{-0.1in}
	\begin{enumerate}[leftmargin=32pt]
	\renewcommand{\labelenumii}{\theenumii}
	\renewcommand{\theenumii}{\theenumi.\arabic{enumii}.}
	\setlength\itemsep{0em}
	\item Center Separation without noise
	\begin{itemize}
	\item Positive result for $\lambda \ge 3$ with no noise.
	\item NP-Hard for $\lambda \le 2$ with no noise.
	\end{itemize}
	\item Center Separation in the presence of noise \begin{itemize}
	\item Lower Bound ($\lambda \le 6$) under arbitrary noise.
	\item Positive result for $\lambda \ge 4$ under sparse noise.
	\item Lower bound ($\lambda < 4$) under sparse noise.
	\end{itemize}
	\end{enumerate}
\item Conclusion
\end{enumerate}


\section{Notation and definition}
\label{sec:Notation}
We are given a data set $\mc X$ drawn from some metric space $(\mb M, d)$ and an integer $k$. A $k$-clustering of $\mc X$ denoted by $\mc C_{\mc X}$ is a partition of $\mc X$ into $k$ disjoints sets. Given points $c_1, \ldots, c_k \in \mb M$, we define the clustering induced by these points (or {\it centers}) such that each $x \in \mc X$ is assigned to its nearest center. In the {\it steiner} setting, the centers can be arbitrary points of the metric space $\mb M$. In the {\it proper} setting, we restrict our centers to be a part of the data set $\mc X$.  In this paper, we will be working in the proper setting.

For any set $\mc A\subseteq \mc X$ with centre $c$, we define the radius of $\mc A$ as $r(\mc A) = \max_{x \in \mc A} d(x, c)$. We will use the notation $\mc C_{\mc A}$ to denote the clustering of a set $\mc A$. 

\vspace{-0.13in}\begin{definition}[$r(\mc C_{\mc X})$ and $t(\mc C_{\mc X})$] Given a clustering $\mc C_{\mc X} = \{C_1, \ldots, C_k\}$ of the set $\mc X$ induced by centers $c_1, \ldots, c_k \in \mb M$. 
\begin{itemize}[nolistsep, noitemsep]
\item $t(\mc{C}_{\mc{X}}) = \min_i |C_i|$
\item $r(\mc{C}_{\mc{X}}) = \max_i r(C_i)$
\end{itemize}
\end{definition}

\vspace{-0.2in}
\begin{definition}[$\mc C_{\mc X}$ restricted to a set] Given $\mc S \subseteq \mc X$ and a clustering $\mc C_{\mc X} = \{C_1, \ldots, C_k\}$ of the set $\mc X$. We define $\mc C_{\mc X}$ restricted to the set $\mc S$ as $\mc C_{{\mc X}|_{\mc S}} = \{C_1 \cap S, \ldots, C_k \cap S\}$. 
\end{definition}

\vspace{-0.2in}
\begin{definition}[$\mc C_{\mc X}$ respects $\mc C_{\mc S}$] Given $\mc S \subseteq \mc X$, clusterings $\mc C_{\mc X} = \{C_1, \ldots, C_k\}$ and $\mc C_{\mc S} = \{S_1, \ldots, S_{k'}\}$. We say that $\mc C_{\mc X}$ respects $\mc C_{\mc S}$ if for all $S_i \in \mc C_{\mc S}$ there exists $C_j \in \mc C_{\mc X}$ such that $S_i \subseteq C_j$ and for all $m \neq i, S_m \cap C_j = \phi$. 
\end{definition}

\begin{definition}[$\mc T$ captures $\mc C_{\mc S}$]Given a hierarchical clustering tree $\mc T$ of $\mc X$ and a clustering $\mc C_{\mc S}$ of $\mc S \subseteq \mc X$. We say that a pruning $\mc P$ respects $\mc C_{\mc S}$ if the clustering $\mc C_{\mc X}^{\mc P}$ corresponding to the pruning respects $\mc C_{\mc S}$. We say that $\mc T$ captures $\mc C_{\mc S}$ if there exists a pruning $\mc P$ which respects $\mc C_{\mc S}$.
\end{definition}

\begin{definition}[$\alpha$-center proximity]
\label{defn:alphacp}
Given a clustering instance $\mc X$ drawn from metric space $(\mb M, d)$ and an integer $k$. A clustering $\mc C_{\mc X} = \{C_1, \ldots, C_k\}$ has $\alpha$-center proximity w.r.t $\mc X$ and $k$ if there exist centers $c_1, \ldots, c_k \in \mb M$  such that the following holds. For all $x \in C_i$ and $i\neq j$, 
\vspace{-0.05in}\begin{align*}
\alpha d(x, c_i) < d(x, c_j)
\end{align*}
Note: In this paper (proper setting), we will restrict the centers $c_i \in \mc X$. 
\end{definition}

\noindent The set $\mc X$ is {\it $\alpha$-center} {\it $t$-min proximal} if there exist a $k$-clustering $\mc C_{\mc X}$  which satisfies $\alpha$-center proximity and $t(\mc C_{\mc X}) = t$. 

\todo\samira Motivate $\eta$-sparse definition.

\begin{definition}[$(\alpha, \eta)$-center proximity]
Given $\mc S \subseteq \mc X$, a clustering $\mc C_{\mc S} = \{S_1, \ldots, S_k\}$ has $(\alpha, \eta)$-center proximity w.r.t $\mc X, \mc S$ and $k$ if there exists centers $s_1, \ldots, s_k \in \mb M$  such that the following holds.
\begin{itemize}[nolistsep, noitemsep]
\label{defn:alphacpnoise}	

\item[$\diamond$] {\bf $\alpha$-centre proximity}: For all $x \in S_i$ and $i\neq j$, $\thinspace\alpha d(x, s_i) < d(x, s_j)$
\item[$\diamond$]{\bf $\eta$-sparse noise}: For any ball $B$ with center $c$, 
\vspace{-0.1in}\begin{align*}
r(B)\leq \eta \thinspace r(\mc{C}_{\mc S}) \implies |B\cap (\mc X\setminus \mc S)| < \frac{t(\mc C_{\mc S})}{2}
\end{align*}
\end{itemize}
Note: In this paper (proper setting), we will restrict the centers $s_i \in \mc X$. 
\end{definition}

\noindent A set $\mc S \subseteq \mc X$ is {\it $(\alpha, \eta)$-proximal} {\it $t$-min nice} if there exist a k-clustering $\mc C_{\mc S}$ which satisfies $(\alpha, \eta)$-center proximity and $t(\mc{C}_{\mc{S}}) = t$. Note that given $(\mc X, d), k$ and $t$ there can be several such $\mc S$ and several clusterings $\mc C_{\mc S}$ which satisfy these conditions.

\begin{definition}[$\lambda$-center separation]
\label{defn:lambdacs}
A clustering $\mc C_{\mc X} = \{C_1, \ldots, C_k\}$ has $\lambda$-center separation w.r.t $\mc X$ and $k$ if there exists centers $c_1, \ldots, c_k \in \mb M$ such that $\mc C_{\mc X}$ is the clustering induced by these centers and the following holds. For all $i\neq j$, 
$$d(c_i, c_j) > \lambda \enspace r(\mc{C}_{\mc{X}})$$
Note: In this paper (proper setting), we will restrict the centers $c_i \in \mc X$.
\end{definition}

\noindent The set $\mc X$ is {\it $\lambda$-separated} {\it $t$-min $r$-max nice} if there exist a clustering $\mc C_{\mc X}$ which satisfies $\lambda$-center separation and $t(\mc{C}_{\mc{X}}) = t$ and $r(\mc{C}_\mc{X}) = r$.

\begin{definition}[$(\lambda, \eta)$-center separation]
Given $\mc S \subseteq \mc X$, a clustering $\mc C_{\mc S}$ has $(\lambda, \eta)$-center separation w.r.t $\mc X, \mc S$ and $k$ if there exists centers $s_1, \ldots, s_k \in \mb M$ and the following holds.

\begin{itemize}[nolistsep, noitemsep]
\label{defn:lambdacsnoise}	

\item[$\diamond$] {\bf $\lambda$-centre separation}: For all $i\neq j$, $\thinspace d(s_i, s_j) > \lambda r({\mc C}_{\mc S})$
\item[$\diamond$]{\bf $\eta$-sparse noise}: For any ball $B$ with center $c$, 
\vspace{-0.1in}\begin{align*}
r(B)\leq \eta \thinspace r(\mc{C}_{\mc S}) \implies |B\cap (\mc X\setminus \mc S)| < \frac{t(\mc C_{\mc S})}{2}
\end{align*}
\end{itemize}
Note: In the setting of this paper, we will restrict the centers $s_i \in \mc X$. 
\end{definition}

\noindent A set $\mc S \subseteq \mc X$ is {\it $(\lambda, \eta)$-separated} {\it $t$-min $r$-max nice} if there exist a clustering $\mc C_{\mc S}$ which satisfies $(\lambda, \eta)$-center separation,  $t({\mc C}_{\mc S}) = t$ and $r({\mc C}_{\mc S}) = r$. Note that given $(\mc X, d), k, r$ and $t$ there can be several such $\mc S$ and several clusterings $\mc C_{\mc S}$ which satisfy these conditions.

We denote a ball of radius $x$ at centre $c$ by $B(c, x)$. We denote by $P_{i}(c)$ a collection of $i$ many points sitting on the same location $c$. If the location is clear from the context, we will use the notation $P_i$.

\section{Center Proximity}
\label{section:cp}
The problem of clustering under $\alpha$-center proximity was first introduced by \cite{awasthi2012center}. They define a data set $\mathcal{X}$ to have $\alpha$-center proximity w.r.t. an objective function $\Phi$, if the optimal clustering of $\mc{X}$ wr.t. $\Phi$ satisfies $\alpha$-center proximity.

Later, \cite{balcan2012clustering} considered the same niceness assumption when $\Phi$ is $k$-median objective function and the clustering instance contains $\epsilon$ fraction of it as noise. In this setting, they construct a hierarchical clustering tree of $\mc{X}$ which captures the optimal clustering $\mc{C}^*$ on the non-noisy points. Their result is restricted to the case that size of noise is bounded by $t(\mc{C}^*)/8$.

In this paper, we consider a natural relaxation of \cite{balcan2012clustering} which aims to capture more realistic clustering instances. We study the limitations of any efficient clustering algorithm in the presence of arbitrary amount of noise. Furthermore, assuming that noise is {\em sparse}, we propose an efficient algorithm which outputs a hierarchical clustering tree $\mc T$ of $\mc X$ that captures all clusterings of all $(\alpha, \eta)$-proximal $t$-min nice subsets of $\mc X$. Here are the main observations of this section:

\begin{itemize}
\item {\it Lower bound with arbitrary noise} - In Section \ref{section:alphaLowerBoundArbitrary}, we show that if the number of noisy points $> \frac{3t}{2}$ and $\alpha \le 2\sqrt{2} + 3 \approx 5.8$, then there is no tree which captures all $\alpha$-center $t$-min proximal subsets of $\mc X$.
\item  {\it Positive result under sparse noise} - In Section \ref{section:positiveResultSparseNoise}, we show that we can overcome this lower bound if we require that the noise is sparse. If $\alpha \ge 2 + \sqrt{7} \approx 4.6$ and $\eta \ge 1$, then Alg.\ref{alg:alphacp} outputs a tree which captures all $(\alpha, \eta)$-proximal $t$-min nice subsets of $\mc X$. Our algorithm runs in time polynomial in $\mc X$. 
\item  {\it Lower bound under sparse noise} - In Section \ref{section:alphaLowerBoundSparse}, we show that if $\alpha \le 2 + \sqrt{3} \approx 3.7$ and $\eta \le 1$, then there does not exist a tree which can capture all $(\alpha, \eta)$-proximal $t$-min nice subsets of $\mc X$.
\end{itemize} 

\subsection{Lower bound under arbitrary noise}
\label{section:alphaLowerBoundArbitrary}

\begin{figure}
\ifdefined\COMPLETE
\else
\documentclass[12pt]{article}
\input{utils} %file containing all the used libraries
\begin{document}
\fi

\def\alph{$2+\sqrt{5}$}
\begin{tikzpicture}[scale=7]

	\coordinate (x) at ($ (-0.02,{sqrt(0)}) $);
	\coordinate (y) at (2.1,0);
	\draw[thick] (x) --(y);

	\node[label=270:$P_{t'}$] at (0.0,0) {$\bullet$};
	\node[label={[yshift=-0.2cm]\scriptsize $\alpha-3$}] at (0.07,0) {};
	\node[label=below:{$P_2$}] at (0.14,0) {$\bullet$};
	\node[label={[yshift=-0.2cm]\scriptsize $2$}] at (0.19,0) {};
	\node[label=270:$P_{t'}$] at (0.24,0) {$\bullet$};
	\node[label={[yshift=-0.2cm]\scriptsize $\alpha-3$}] at (0.31,0) {};
	\node[label=270:$P_2$] at (0.38,0) {$\bullet$};
	\node[label={[yshift=-0.2cm]\scriptsize $2$}] at (0.43,0) {};
	\node[label=270:$P_{t'}$] at (0.48,0) {$\bullet$};
	\node[label={[yshift=-0.2cm]\scriptsize $\alpha-3$}] at (0.55,0) {};
	\node[label=270:$P_2$] at (0.62,0) {$\bullet$};
	\node[label={[yshift=-0.2cm]\scriptsize $2$}] at (0.67,0) {};
	\node[label=270:$P_{t'}$] at (0.72,0) {$\bullet$};	
	\node[label={[yshift=-0.2cm]\scriptsize $2$}] at (0.77,0) {};
	\node[label=270:$P_2$] at (0.82,0) {$\bullet$};	
	\node[label={[yshift=-0.2cm]\scriptsize $\alpha-3$}] at (0.89,0) {};
	\node[label=270:$P_{t'}$](B) at (0.96,0) {$\bullet$};	
	\node at (1.89,0.03) {$\ldots\ldots\ldots$};	
	\node[label=270:$P_2$] at (1.06,0) {$\bullet$};	
	\node[label=270:$P_{t'}$] at (1.2,0) {$\bullet$};	
	\node[label={[yshift=-0.2cm]\scriptsize $2$}] at (1.25,0) {};
	\node[label=270:$P_2$] at (1.3,0) {$\bullet$};	
	\node[label={[yshift=-0.2cm]\scriptsize $\alpha-3$}] at (1.37,0) {};
	\node[label=270:$P_{t'}$] at (1.44,0) {$\bullet$};	
	\node[label={[yshift=-0.2cm]\scriptsize $2$}] at (1.05,0) {};
	\node[label={[yshift=-0.2cm]\scriptsize $\alpha-3$}] at (1.15,0) {};
	\node[label={[yshift=-0.2cm]\scriptsize $10\alpha$}] at (1.59,0) {};
	\node[label=270:$B_3$](B) at (1.74,0) {$\bullet$};	
	\node at (1.89,0.03) {$\ldots\ldots\ldots$};	
	\node[label=270:$B_k$](p3end) at (2.04,0) {$\bullet$};	
	\node(A) at (1.89,0.1) {\small $k-2$};	
	
	\node[label=270:$B_1$] at (0.12,-0.1) {};	
	\node[label=270:$B_1'$] at (0.36,-0.1) {};	
	\node[label=270:$B_1''$] at (0.6,-0.1) {};	
	\node[label=270:$B_2$] at (0.84,-0.1) {};	
	\node[label=270:$B_2'$] at (1.08,-0.1) {};
	\node[label=270:$B_2''$] at (1.32,-0.1) {};	
	
	\draw (0.12,0) circle (0.12);
	\draw[dashed, blue] (0.36,0) circle (0.12);
	\draw[dotted, green, thick] (0.6,0) circle (0.12);
	\draw (0.84,0) circle (0.12);
	\draw[dashed, blue] (1.08,0) circle (0.12);
	\draw[green, thick, dotted] (1.32,0) circle (0.12);
	\draw[decorate,decoration={brace,amplitude=5pt}]
	(1.74, 0.05) -- (2.04, 0.05);
	%\draw (A) -- (1.74, 0.06);
\end{tikzpicture}


\ifdefined\COMPLETE
\else
\end{document}
\fi

\caption{$\mc X \subseteq \mathbb{R}$ such that no algorithm can capture all the $\alpha$-proximal $t$-min nice subsets of $\mc X$. } 
\label{fig:nosparsealg}
\end{figure}

Fig.\ref{fig:nosparsealg} shows an instance of a set $\mc X$ with $\alpha$-proximal $t$-min nice subsets $\mc S, \mc S', \mc S'' \subset \mc X$. We show that these subsets are ``incompatible" i.e., there is no  hierarchical clustering tree that can capture all the clusterings of these subsets that satisfy $\alpha$-center proximity. To prove this, we will show that the clusterings $\mc C_{\mc S} = \{B_1, B_2, B_3, \ldots, B_k\}$, $\mc C_{\mc S'} = \{B_1', B_2', B_3, \ldots, B_k\}$ and $\mc C_{\mc S''} = \{B_1'', B_2'', B_3, \ldots, B_k\}$ are incompatible with one another.

\begin{theorem}
\label{thm:nosparsealg}
Given the number of clusters $k$ and parameters $\alpha$ and $t$. There exists a clustering instance $(\mc X, d)$ such that any clustering tree $\mc T$ of $\mc X$ has  the following property. There exists $\mc S \subseteq \mc X$ and there exists clustering $\mc C_{\mc S} = \{S_1, \ldots, S_k\}$ which satisfies $\alpha$-center proximity w.r.t $\mc S$ and $k$ and $ t(\mc C_{\mc S}) = t \ge 2$ such that the following holds. 

If $\alpha < 2\sqrt 2 + 3$ and $|\mc X \setminus \mc S| \ge \frac{3t}{2}+5$, then $\mc T$ doesn't capture $\mc C_{\mc S}$.
\end{theorem}

\begin{proof}
Define $\mc X \subseteq \mathbb{R}$ as follows. Let $t' = \frac{t}{2}-1$. Let $O_1 = \cup_{i=0}^6 P_{t'}(i\alpha-i)$ and $O_2 = \cup_{i=1}^3 P_2(i\alpha-i-2) \cup_{i=3}^5 P_2(i\alpha-i+2)$. For $3\le i\le k$, define $B_i = B(16\alpha-6+i\alpha, 0)$. Now, let $\mc X = \{O_1 \cup O_2 \cup B_3 \cup \ldots \cup B_k\}$. The set is also shown in Fig. \ref{fig:nosparsealg}.

Now, let $B_1 = \{P_{t'}(0) \thinspace\cup\thinspace  P_{2}(\alpha-3) \thinspace\cup\thinspace P_{t'}(\alpha-1) \}$ and $B_2 = \{\ P_{t'}(3\alpha-3) \thinspace\cup\thinspace P_{2}(3\alpha-1)\thinspace\cup\thinspace P_{t'}(4\alpha-4)\}$. Also, let $B_1' = \{P_{t'}(\alpha-1) \thinspace\cup\thinspace  P_{2}(2\alpha-4) \thinspace\cup\thinspace P_{t'}(2\alpha-2) \}$ and $B_2' = \{\ P_{t'}(4\alpha-4) \thinspace\cup\thinspace P_{2}(4\alpha-2)\thinspace\cup\thinspace P_{t'}(5\alpha-5)\}$. Also, let $B_1'' = \{P_{t'}(2\alpha-2) \thinspace\cup\thinspace  P_{2}(3\alpha-5) \thinspace\cup\thinspace P_{t'}(3\alpha-3) \}$ and $B_2'' = \{\ P_{t'}(5\alpha-5) \thinspace\cup\thinspace P_{2}(5\alpha-3)\thinspace\cup\thinspace P_{t'}(6\alpha-6)\}$. 

Now, define $\mc S = \{B_1 \cup B_2 \cup B_3\ldots \cup B_k\} \subseteq \mc X$, $\mc S' = \{B_1' \cup B_2' \cup B_3\ldots \cup B_k\} \subseteq \mc X$ and $\mc S'' = \{B_1'' \cup B_2'' \cup B_3\ldots \cup B_k\} \subseteq \mc X$. It is easy to see that for $\alpha \le 2\sqrt{2} + 3$, all of $\mc S$, $\mc S'$ and $\mc S''$ are $\alpha$-center $t$-min nice and the clusterings $\mc C_{\mc S} = \{B_1, B_2, B_3, \ldots, B_k\}$, $\mc C_{\mc S'} = \{B_1', B_2', B_3, \ldots, B_k\}$ and $\mc C_{\mc S''} = \{B_1'', B_2'', B_3, \ldots, B_k\}$ satisfy $\alpha$-center proximity. However, we will show that no algorithm can output a tree which respects $\mc C_{\mc S}$, $\mc C_{\mc S'}$ and $\mc C_{\mc S''}$.

For the sake of contradiction assume that there exists a tree $T$ and prunings $P$, $P'$ and $P''$ such that $P$ respects $\mc C_{\mc S}$, $P'$ respects $\mc C_{\mc S'}$ and $P''$ respects $\mc C_{\mc S''}$. By reasoning similar to Thm. \ref{thm:noalgalphacp}, we get that there exists a node $N_1$ which contains $B_1$, $B_1'$ and $B_1''$ and doesn't any point from one of $B_2'$ or $B_2''$. By symmetry, there exists node $N_2$ which contains $B_2$, $B_2'$ and $B_2''$ and doesn't contain any point from $B_1$ or $B_1'$. Observe that $N_1 \cap N_2 \neq \phi$ hence either $N_1 \subseteq N_2$ or $N_2 \subseteq N_1$. WLOG, let $N_1 \subseteq N_2$. Then, we get that $N_2$ doesn't contain any point from $B_1$ or $B_1'$ and contains both $B_1$ and $B_1'$. This leads to a contradiction. Hence, there exists no tree which contains prunings which respect all of $\mc C_{\mc S}$, $\mc C_{\mc S'}$ and $\mc C_{\mc S''}$.
\end{proof}

\subsection{Positive result under sparse noise}
\label{section:positiveResultSparseNoise}

Given a clustering instance $(\mc X, d)$ and a parameter $t$, we propose an efficient algorithm which outputs a hierarchical clustering tree $\mc T$ of $\mc X$ with the following property. For every $k$, for every $\mc S \subseteq \mc X$ which is $(\alpha, \eta)$-proximal $t$-min nice and for every $k$-clustering $\mc C_{\mc S}$ which satisfies $(\alpha, \eta)$-center proximity, $\mc T$ captures $\mc C_{\mc S}$. It is important to note that our algorithm only knows $\mc X$ and has no knowledge of the set $\mc S$.


Our algorithm has a linkage based structure similar to \cite{balcan2012clustering}. However, our method benefits from a novel {\it sparse distance condition}. We introduce the algorithm in Alg.\ref{alg:alphacp} and prove it's efficiency and correctness in Theorem. \ref{thm:algcptime} and Theorem.\ref{thm:alphacpnoise} respectively. 

\begin{definition}[Sparse distance condition]
	 Given a clustering $\mc C = \{C_1,\ldots,C_k\}$ of the set $\mc X$ and a ball $B \subseteq \mc X$. Define $Y_B^{\mc C} := \{C_i \in \mc C_{\mc X} : C_i \subseteq B \text{ or } |B \cap C_i| \ge t/2\}$. 
We say that the ball $B$ satisfies the sparse distance condition w.r.t clustering $\mc C$ when the following holds.
\begin{itemize}[noitemsep, leftmargin=*]
\item $|B| \ge t$.
\item For any $C_i \in \mc C$, if $C_i \cap B \neq \phi$, then $C_i \in Y_B^{\mc C}$.
\end{itemize}
\end{definition}

Intuitively, Alg.\ref{alg:alphacp} works as follows. It maintains a clustering $\mc C^{(l)}$, which is initialized so that each point is in its own cluster. It then goes over all pairs of points $p, q$ in increasing order of their distance $d(p, q)$. If $B(p, d(p,q))$ satisfies the sparse distance condition w.r.t $\mc C^{(l)}$, then it merges all the clusters which intersect with this ball into a single cluster and updates $\mc C^{^(l)}$. Furthermore, the algorithm builds a tree with the nodes corresponding to the merges performed so far. We will show that for all $\mc S \subseteq \mc X$ which are $(\alpha, \eta)$-proximal $t$-min nice and for all clusterings $\mc C_S$ which have $(\alpha, \eta)$-center proximity, Alg. \ref{alg:alphacp} outputs a tree which captures $\mc C_S$.

	
\begin{algorithm}
	\SetAlgoLined
	\KwIn{$(\mc X, d)$ and $t$}
	\KwOut{A hierarchical clustering tree $T$ of $\mc X$.}
	
	\vspace{0.1in}Let $\mc C^{(l)}$ denote the clustering $\mc X$ after $l$ merge steps have been performed. Initialize $\mc C^{(0)}$ so that all points are in their own cluster. That is, $\mc C^{(0)} = \{ \{x\}: x \in \mc X\}$.
	
	Go over all pairs of points $p, q$ in increasing order of the distance $d(p, q)$. 
	\begin{itemize}[noitemsep]
	\renewcommand\labelitemi{}
	\item If $B = B(p, d(p, q))$ satisfies the sparse distance condition then
		\begin{itemize}[noitemsep]
		\renewcommand\labelitemii{}
		\item $\mc C^{(l+1)} = \mc C^{(l)}$
		\item Merge all the clusters in $Y_B^{C^{(l)}}$ into a single cluster.
		\item Update $\mc C^{(l+1)}$.
		\end{itemize}
	\end{itemize}
	
	Output clustering tree $T$. The leaves of $T$ are the points in dataset $\mc X$. The internal nodes correspond to the merges performed.
\caption{Alg. for $(\alpha, \eta)$-center proximity with parameter $t$}
\label{alg:alphacp}
\end{algorithm}


\begin{theorem}
\label{thm:alphacpnoise}
Given a clustering instance $(\mc X, d)$, the number of clusters $k$ and a parameter $t$. Alg. \ref{alg:alphacp} outputs a tree $T$ with the following property. For all $k$, for all $\mc S \subseteq \mc X$ and for all $k$-clusterings $\mc C^*_{\mc S} = \{S_1^*, \ldots, S_k^*\}$ which satisfy $(2+\sqrt{7}, 1)$-center proximity w.r.t $\mc X, \mc S$ and $k$ such that $t(\mc C_{\mc S}^*) = t \ge 2$ the following holds. 

For every $1\le i \le k$, there exists a node $N_i$ in the tree $T$ such that $S_i^* \subseteq N_i$ and for $j \neq i$, $S_j^* \cap N_i = \phi$ . That is, $N_i$ contains points from only one of the good clusters $S_i^*$.
\end{theorem}

\begin{proof}
Fix any $\mc S \subseteq \mc X$. Let $\mc C^*_S = \{S_1^*, \ldots, S_k^*\}$ be a clustering of $\mc S$ such that $t(\mc C_{\mc S}^*) = t$ and $\mc C^*_S$ has $(\alpha, \eta)$-center proximity. Denote by $r_i := r(S_i^*)$. Let $\mc C^{(l)} = \{C_1, \ldots, C_{k'}\}$ be the clustering of $\mc X$ after $l$ merge steps have been performed (as defined in Alg.\ref{alg:alphacp}). Let $p, q \in \mc X$ be the pair of points being considered at the $(l+1)^{th}$ step. Let's denote $B = B(p, d(p, q))$. Note that whenever $B$ satisfies the sparse-distance condition, all the clusters in $Y_{B}^{{\mc C}^{(l)}}$ are merged together and the clustering $\mc C^{(l+1)}$ is updated. Throughout the proof, we will denote by $S_i^* \in \mc C^*_S$ the clusters of the set $\mc S$ (also called ``good" clusters) and by $C_i \in \mc C^{(l)}$ the clusters of the set $\mc X$ obtained after $l$ merge operations.

\noindent We will prove the theorem by proving two key facts.

\begin{enumerate}[nolistsep, noitemsep, label=\textbf{F.\arabic*}]
\renewcommand\labelitemi{$\diamond$}
\item \label{fact:1} If the algorithm merges points from a good cluster $S_i^*$ with points from some other good cluster,  then at this step the distance being considered $d = d(p,q) > r_i$.	
\item \label{fact:2} When the algorithm considers the distance $d = r_i$, it merges all points from $S_i^*$ (and possibly points from $\mc X\setminus \mc S$) into a single cluster $C_i$. Hence, there exists a node in the tree $N_i$ which contains all the points from $S_i^*$ and no points from any other good cluster $S_j^*$. 	
\end{enumerate}
Note that the theorem follows from these two facts. We now state both of these facts formally below and prove them.

\noindent\textit{\underline{Proof of Fact \ref{fact:1}}}\\
Let $\mc C^{(l)} = \{C_1, \ldots, C_{k'}\}$ be the current clustering of $\mc X$. Let $l+1$ be the first merge step which merges points from the good cluster $S_i^*$ with points from some other good cluster. Let $p, q \in \mc X$ be the pair of points being considered at this step and $B = B(p, d(p, q))$ the ball that satisfies the sparse distance condition at this merge step. Denote by $Y = Y_{B}^{C^{(l)}}$. We need to show that $d(p, q) > r_i$. To prove this, we need Claim \ref{claim:fromBothCluster} below. 
\begin{claim}
\label{claim:fromBothCluster}
Let $p, q \in \mc X$ and $B$, $Y$, $S_i^*$ and $C^{(l)}$ be as defined above. If $d(p, q) \le r,$ then $B \cap S_i^* \neq \phi$ and there exists $n \neq i$ such that $B \cap S_n^* \neq \phi$.
\end{claim}
\vspace{-0.1in} Observe that $l+1$ is the first step which merges points from $S_i^*$ with some other good cluster. Hence, there exists a cluster $C_i \in Y$ such that $C_i\cap S_i^*  \neq \phi$ and for all $n \neq i$, $C_i \cap S_n^* = \phi$. Also, there exists cluster $C_j \in Y$ such that $C_j \cap S_j^* \neq \phi$ for some $S_j^*$ and $C_j \cap S_i^* = \phi$.

$C_i \in Y$. Hence, $C_i \subseteq B$ or $|C_i \cap B| \ge t/2$. In the first case, if $C_i \subseteq B$ then $B \cap S_i* \neq \phi$ by set inclusion. In the second case assume that $|C_i \cap B| \ge t/2$. For the sake of contradiction, assume that $B$ contains no points from $S_i^*$. That is, $B \cap C_i \subseteq C_i \setminus S_i^* \subseteq \mc X \setminus \mc S$. This implies that $B \cap C_i \subseteq B \cap \{\mc X \setminus \mc S\}$. Hence, $|B\cap \{\mc X \setminus \mc S\}| \ge |B \cap C_i| \ge t/2$. This contradicts the sparse noise assumption in Defn. \ref{defn:alphacpnoise}. Hence, $B \cap S_i^* \neq \phi$.

$C_j \in Y$. Hence, $C_j \subseteq B$ or $|C_j \cap B| \ge t/2$. In the first case, if $C_j \subseteq B$ then $B \cap S_j^* \neq \phi$ by set inclusion. In the second case assume that $|C_j \cap B| \ge t/2$. For the sake of contradiction, assume that for all $n \neq i$, $B$ contains no points from $S_n^*$. That is, $B \cap C_j \subseteq C_j \setminus (\cup_{n \neq i} S_n^*) \subseteq \mc X \setminus \mc S$. This implies that $B \cap C_j \subseteq B \cap \{\mc X \setminus \mc S\}$. Hence, $|B\cap \{\mc X \setminus \mc S\}| \ge |B \cap C_j| \ge t/2$. This contradicts the sparse noise assumption in Defn. \ref{defn:alphacpnoise}. Hence, there exists $S_n^* \neq S_i^*$ such that $B \cap S_n^* \neq \phi$.

\begin{claim}
\label{claim:maxrirj}
Let the framework be as given in Claim \ref{claim:fromBothCluster}. Then, $d(p, q) > r_i$.
\end{claim}

\vspace{-0.1in} If $d(p, q) > r$, then the claim follows trivially. We assume that $d(p, q) \le r$. From claim \ref{claim:fromBothCluster}, $B$ contains $p_i \in S_i^*$ and $p_j \in S_j^*$. Let $r_i = d(c_i, q_i)$ for some $q_i \in S_i^*$.
\begin{align*}
d(c_i, q_i) &< \frac{1}{\alpha} d(q_i, c_j) \le \frac{1}{\alpha} \bigg[ d(p_j, c_j) + d(p_i, p_j) + d(p_i, q_i)\bigg] < \frac{1}{\alpha} \bigg[ \frac{1}{\alpha}d(p_j, c_i) + d(p_i, p_j) + 2d(c_i, q_i)\bigg]\\
& < \frac{1}{\alpha} \bigg[ \frac{1}{\alpha}d(p_i, p_j) + \frac{1}{\alpha}d(c_i, q_i) + d(p_i, p_j) + 2d(c_i, q_i)\bigg]
\end{align*}
This implies that $(\alpha^2 - 2\alpha - 1)d(q_i, c_i) < (\alpha + 1) d(p_i, p_j)$. For $\alpha \ge 2 + \sqrt 7$, this implies that $d(c_i, q_i) < d(p_i, p_j)/2$. Now, using triangle inequality, we get that $d(c_i, q_i) < d(p_i, p_j)/2 \le \frac{1}{2}[d(p, p_i) + d(p, p_j)] < d(p, q)$.\\

\noindent\textit{\underline{Proof of Fact \ref{fact:2}
}}\\
Let $\mc C^{(l)} = \{C_1, \ldots, C_{k'}\}$ be the current clustering of $\mc X$. Let $l+1$ be the merge step when $p = s_i$ and $q = q_i$ such that $d(s_i, q_i) = r_i$. We will prove that the ball $B = B(s_i, q_i)$ satisfies the sparse-distance condition. Hence, all the points from the good cluster $S_i^*$ will be merged together. Note that this step merges all the clusters in $Y = Y_B^{C^{(l)}}$. Hence to complete the proof, we also show that all the clusters in $Y$ don't intersect with any other good cluster $S_j^*$. We state this formally below and then prove it.

\begin{claim}
%\vspace{-0.1in}
\label{claim:dciqi}
Let $s_i$, $q_i$, $r_i$, $B$ and $Y$ be as defined above. Then, $B$ satisfies the sparse distance condition and for all $C \in Y$, for all $j \neq i, C \cap S_j^* = \phi$.
\end{claim}

\vspace{-0.1in} Denote by $B = B(s_i, q_i)$. $|B| = |S_i^*| \ge t$. Observe that, for all $C \in \mc C^{(l)}$, $|C| = 1$ or $|C| \ge t$. We need to prove two statements. If $C \cap B \neq \phi$, then $C \in Y$ and $C \cap S_j^* = \phi$. 

\begin{itemize}[nolistsep]
\item Case 1. $|C| = 1$. If $C \cap B \neq \phi \implies C \subseteq B = S_i^*$. Hence, $C \in Y$ and for all $j \neq i$, $C \cap S_j^* = \phi$

\item Case 2. $|C|\ge t$. $C \cap B \neq \phi$. Let $h(C)$ denote the height of the cluster in the tree $T$. Now, we consider two subcases.
\begin{itemize}
\renewcommand\labelitemii{$\circ$}
\item Case 2.1. $h(C) = 1$. In this case, there exists a ball $B'$ such that $B' = C$. We know that $r(B') \le r_i \le r$. Hence using Claim \ref{claim:maxrirj}, we get that for all $j \neq i$, $B' \cap S_j^* = \phi$. Thus, $|B'\setminus S_i^*| \le t/2 \implies |B\cap C| = |C| - |C\setminus B| = |C| - |B'\setminus S_i^*| \ge t/2$. Hence, $C \in Y$.

\item Case 2.2. $h(C) > 1$. Then there exists some $C'$ such that $h(C') = 1$ and $C' \subset C$. Now, using set inclusion and the result from the first case, we get that $|B\cap C| \ge |B\cap C'| \ge t/2$. Hence, $C \in Y$. Using Claim \ref{claim:maxrirj}, we get that for all $j \neq i$, $C \cap S_j^* = \phi$.
\end{itemize} 
\end{itemize}
\end{proof}

\begin{theorem}
\label{thm:algcptime}
Given clustering instance $(\mc X, d)$ and $t$. Algorithm \ref{alg:alphacp} runs in time which is polynomial in $|\mc X|$.
\end{theorem}

\begin{proof}
Let $n = |\mc X|$. Let $\mc C^{(l)} =\{C_1, \ldots, C_{k'}\}$ denote the current clustering of $\mc X$ as defined in Alg. \ref{alg:alphacp}. Observe that given the ball $B = B(p, d(p, q))$, for any cluster $C_i \in \mc C^{(l)}$, checking if $C_i \in Y_B^{C^{(l)}}$ takes time O($|C_i|$). Doing this for all clusters takes $O(n)$ time. The algorithm examines all the pairs of points and hence runs in $O(n^3)$ time.
\end{proof}

Next, we make a small note on the conditions on $\alpha$ and $\eta$ under which the tree outputted by Alg. \ref{alg:alphacp} doesn't capture all $(\alpha, \eta)$-proximal $t$-min nice subsets of $\mc X$. We construct a set $\mc X$ (Fig. \ref{fig:algAlphacp}) and show that if $\alpha < 2 + \sqrt{5}$ and $\eta \le 1$ then Alg. \ref{alg:alphacp} can `split' some of the good cluster. The details and the proof can be found in Appendix \ref{appendix:sectiontr}.
 


\subsection{Lower bound under sparse noise}
\label{section:alphaLowerBoundSparse}

We showed that for $\alpha \ge 2 + \sqrt{7}$ and $\eta \ge 1$, Alg. \ref{alg:alphacp} outputs a tree which captures all $(\alpha, \eta)$-proximal $t$-min nice subsets of $\mc X$ (Theorem.~\ref{thm:alphacpnoise}). In this section, we will show that if $\alpha \le 2 + \sqrt{3} \approx 3.7$ and $\eta \le 1$, then there is no tree (and hence no algorithm can output a tree) which captures all $(\alpha, \eta)$-proximal $t$-min nice subsets of $\mc X$.

\begin{figure}
\ifdefined\COMPLETE
\else
\documentclass[12pt]{article}
\input{utils} %file containing all the used libraries
\begin{document}
\fi

\def\alph{$2+\sqrt{5}$}
\begin{tikzpicture}[scale=7]

	\coordinate (x) at ($ (-0.20,{sqrt(0)}) $);
	\coordinate (y) at (2.0,0);
	\draw[thick] (x) --(y);

	\node[label=270:$P_{t_2}$] at (-0.20,0) {$\bullet$};
	\node[label={[yshift=-0.2cm]\scriptsize $\alpha-2$}] at (-0.1,0) {};
	\node[label=270:$P_1$] at (0.0,0) {$\bullet$};
	\node[label={[yshift=-0.2cm]\scriptsize $1$}] at (0.06,0) {};
	\node[label=below:{$P_{t_1}$}] at (0.12,0) {$\bullet$};
	\node[label={[yshift=-0.2cm]\scriptsize $\alpha-2$}] at (0.22,0) {};
	\node[label=270:$P_1$] at (0.32,0) {$\bullet$};
	\node[label={[yshift=-0.2cm]\scriptsize $1$}] at (0.38,0) {};
	\node[label=270:$P_{t_2}$] at (0.44,0) {$\bullet$};
	\node[label={[yshift=-0.2cm]\scriptsize $1$}] at (0.5,0) {};
	\node[label=270:$P_1$] at (0.56,0) {$\bullet$};
	\node[label={[yshift=-0.2cm]\scriptsize $\alpha-2$}] at (0.66,0) {};
	\node[label=270:$P_{t_1}$] at (0.76,0) {$\bullet$};
	\node[label={[yshift=-0.2cm]\scriptsize $1$}] at (0.82,0) {};
	\node[label=270:$P_1$] at (0.88,0) {$\bullet$};	
	\node[label={[yshift=-0.2cm]\scriptsize $\alpha-2$}] at (0.98,0) {};
	\node[label=270:$P_{t_2}$] at (1.08,0) {$\bullet$};	
	\node[label={[yshift=-0.2cm]\scriptsize $10\alpha$}] at (1.29,0) {};
	\node[label=270:$B_3$](p3start) at (1.45,0) {$\bullet$};	
	\node[label={[yshift=-0.2cm]\scriptsize $\alpha$}] at (1.57,0) {};
	\node[label=270:$B_4$](B) at (1.65,0) {$\bullet$};	
	\node at (1.79,0.03) {$\ldots\ldots\ldots$};	
	\node[label=270:$B_k$](p3end) at (1.95,0) {$\bullet$};	
	\node(A) at (1.79,0.1) {\small $k-2$};	
	
	\node[label=270:$B_1$] at (-0.05,-0.15) {};	
	\node[label=270:$B_1'$] at (0.27,-0.15) {};	
	\node[label=270:$B_2$] at (0.6,-0.15) {};	
	\node[label=270:$B_2'$] at (0.92,-0.15) {};
	
	\draw (-0.04,0) circle (0.16) ;
	\draw[dashed] (0.28,0) circle (0.16);
	\draw (0.6,0) circle (0.16cm);
	\draw[dashed] (0.92, 0.0) circle (0.16) ;
	\draw[decorate,decoration={brace,amplitude=5pt}]
	(1.45, 0.05) -- (1.95, 0.05);
	%\draw (A) -- (1.74, 0.06);
\end{tikzpicture}


\ifdefined\COMPLETE
\else
\end{document}
\fi

\caption{$\mc X \subseteq \mathbb{R}$ such that no algorithm can capture all the $(\alpha, \eta)$-proximal $t$-min nice subsets of $\mc X$.}
\label{fig:noalgalphacp}
\end{figure}

Fig. \ref{fig:noalgalphacp} shows an instance of a set $\mc X$ with $(\alpha, \eta)$-proximal $t$-min nice subsets $\mc S, \mc S', \mc S'' \subset \mc X$. We show that these subsets are ``incompatible" i.e., there is no  hierarchical clustering tree that can capture all the clusterings of these subsets that satisfy $\alpha$-center proximity. To prove this, we will show that the clusterings $\mc C_{\mc S} = \{B_1, B_2, B_3, \ldots, B_k\}$ and $\mc C_{\mc S'} = \{B_1', B_2', B_3, \ldots, B_k\}$ are incompatible with one another.

\begin{theorem}
\label{thm:noalgalphacp}
Given the number of clusters $k$ and parameters $\alpha$, $\eta$ and $t$. There exists a clustering instance $(\mc X, d)$ such that any clustering tree $\mc T$ of $\mc X$ has  the following property. There exists $\mc S \subseteq \mc X$ and there exists clustering $\mc C_{\mc S} = \{S_1, \ldots, S_k\}$ which satisfies $(\alpha, \eta)$-center proximity w.r.t $\mc X, \mc S$ and $k$ and $ t(\mc S) = t \ge 2$ such that the following holds. 

If $\alpha \le 2 + \sqrt 3$ and $\eta \le 1$ then $\mc T$ doesn't capture $\mc C_{\mc S}$.
\end{theorem}

\begin{proof}
Define $\mc X \subseteq \mathbb{R}$ as follows. Let $t_1 = \frac{t}{2}+1$ and $t_2 = \frac{t}{2}-2$. Let $O_1 = P_{t_1}(\alpha-1)\cup P_{t_1}(3\alpha-3)$ and $O_2 = \{P_{1}(\alpha-2) \thinspace\cup\thinspace P_{1}(2\alpha-3) \thinspace\cup\thinspace P_{1}(2\alpha-1) \thinspace\cup\thinspace P_{1}(3\alpha-2)\}$ and $O_3 = P_{t_2}(0)\cup P_{t_2}(2\alpha-2)\cup P_{t_2}(4\alpha-4)$. For $3\le i\le k$, define $B_i = P_t(14\alpha-4+i\alpha)$. Now, let $\mc X = \{O_1 \cup O_2 \cup O_3\cup B_3 \cup \ldots \cup B_k\}$. The set is also shown in Fig. \ref{fig:noalgalphacp}.

Now, let $B_1 = \{P_{t_2}(0) \thinspace\cup\thinspace  P_{1}(\alpha-2) \thinspace\cup\thinspace P_{t_1}(\alpha-1) \}$ and $B_2 = \{\ P_{t_2}(2\alpha-2) \thinspace\cup\thinspace P_{1}(2\alpha-1)\thinspace\cup\thinspace P_{t_1}(3\alpha-3)\}$. Also, let $B_1' = \{P_{t_1}(\alpha-1) \thinspace\cup\thinspace  P_{1}(2\alpha-3) \thinspace\cup\thinspace P_{t_2}(2\alpha-2) \}$ and $B_2' = \{\ P_{t_1}(3\alpha-3) \thinspace\cup\thinspace P_{1}(3\alpha-2)\thinspace\cup\thinspace P_{t_2}(4\alpha-4)\}$. 

Now, define $\mc S = \{B_1 \cup B_2 \cup B_3\ldots \cup B_k\} \subseteq \mc X$ and $\mc S' = \{B_1' \cup B_2' \cup B_3\ldots \cup B_k\} \subseteq \mc X$. It is easy to see that for $\alpha \le 2+\sqrt{3}$, both $\mc S$ and $\mc S'$ are $(\alpha, 1)$-center $t$-min nice and the clusterings $\mc C_{\mc S} = \{B_1, B_2, B_3, \ldots, B_k\}$ and $\mc C_{\mc S'} = \{\ B_1', B_2', B_3, \ldots, B_k\}$ satisfy $(\alpha, 1)$-center proximity. However, we will show that no algorithm can output a tree which respects both $\mc C_{\mc S}$ and $\mc C_{\mc S'}$.

For the sake of contradiction assume that there exists a tree $T$ and prunings $P$ and $P'$ such that $P$ respects $\mc C_{\mc S}$ and $P'$ respects $\mc C_{\mc S'}$. Hence, there exists a node $N_1$ such that $B_1 \subseteq N_1$ and $N_1 \cap B_2 = \phi$ and there exists $N_2$ such that $B_2 \subseteq N_2$ and $N_2 \cap B_1 = \phi$. Also, there exists node $N_1'$ such that such that $B_1' \subseteq N_1'$ and $N_1' \cap B_2' = \phi$. Now, observe that $B_1'$ contains points from both $B_1$ and $B_2$. Hence, both $N_1 \subseteq N_1'$ and $N_2 \subseteq N_1'$. Thus, $P_{t'}(3\alpha-3)\in B_2'\cap B_2 \subseteq B_2' \cap N_2 \subseteq B_2' \cap N_1'$. This contradicts the fact that $N_1' \cap B_2' = \phi$. Hence, prunings $P$ and $P'$ can not exist.
\end{proof}



\section{Center Separation}
The problem of noise robust clustering under center separation (Def.~\ref{defn:lambdacs}) was first introduced by \cite{ben2014clustering}. Their goal is to robustify any objective based clustering algorithm to noise.  In this direction, they define a data set $\mc X$ to be clusterable, if the optimal solution of any objective based clustering algorithm on $\mc X$ has $\lambda$-center sepration. Given such an input, they propose a paradigm which converts any objective-based clustering algorithm into a clustering algorithm which is robust to small amount of noise.

Although the framework of \cite{ben2014clustering} works for any objective-based clustering algorithm, it compels a strong restriction on noise and separation of the data. For example, when the size of the noise is $\frac{5}{100}|\mc X|$, to obtain a robustified version of $2$-median, their algorithm needs to get an input $\mc X$ which has a clustering ${\mc C}_{\mc X}$ with $(\lambda > 10)$-center separation.

\samira\wip

 
\subsection{Center Separation without noise}
\todo\shrinu

\subsubsection{Positive result under no noise}
We are given a clustering instance $(\mc X, d)$. Our goal is to output a hierarchical clustering tree $T$ of $\mc X$ which has the following property. For every $k$ and for every $k$-clustering $\mc C_{\mc X}$ of the set $\mc X$ which satisfies $\lambda$-center separation, there exists a pruning $\mc P$ of the tree such that $\mc C_{\mc X}^{\mc P}$ (the clustering of the set $\mc X$ corresponding to the pruning $\mc P$) equals $\mc C_{\mc X}$. 

Our algorithm (Alg. \ref{alg:lambdacsNoNoise}) uses single-linkage to build a hierarchical clustering tree of $\mc X$. We will show that when $\lambda \ge 3$ our algorithm achieves the above mentioned goal. Before we prove the result, we need to introduce a bit of notation. Given $\mc X$ drawn from some metric space $(M, d)$, for $A, B \subseteq \mc X$ we define $d_{min}(A, B) = \min \{d(a, b): a \in A$ and $b \in B\}$.
\begin{definition}[Strong stability \cite{balcan2008discriminative}]
Given a clustering instance $\mc X$ drawn from some metric space $(\mb M, d)$ and a clustering $\mc C_{\mc X} = \{C_1, \ldots, C_k\}$ of $\mc X$. We say that $d$ has strong stability w.r.t $\mc C_{\mc X}$ when for all $C_i, C_j \in \mc C_{\mc X}$ and for all $A \subset C_i$, $A' \subseteq C_j$, we have that 
$$d_{min}(A, C_i\setminus A) < d_{min}(A, A')$$
\end{definition}

If a clustering instance satisfies strong stability then single-linkage produces a tree such that the gound-truth clustering is a pruning of the tree (Theorem. 8 in \cite{balcan2008discriminative}). We will prove that any clustering instance which satisfies $\lambda$-center separation for $\lambda \ge 3$ also satisfies strong stability. Hence, the claim follows.

\begin{algorithm}[t]
	\SetAlgoLined
	\KwIn{$(\mc X, d)$}
	\KwOut{A hierarchical clustering tree $T$ of $\mc X$.}
	
	\vspace{2mm} Initialize the clustering so that each point is in its own cluster.
	
	Run single-linkage till only a single cluster remains. Output clustering tree $T$.

\caption{Alg. for $\lambda$-center separation}
\label{alg:lambdacsNoNoise}
\end{algorithm}

\begin{theorem}
Given a clustering instance $(X , d)$ and the number of clusters $k$. Alg. \ref{alg:lambdacsNoNoise} outputs a tree $T$ with the following property. For all $k$ and for all $k$-clusterings $\mc C_{\mc X}^* = \{C_1^*, \ldots, C_k^* \}$ of the set $\mc X$ which satisfy $\lambda$-center separation w.r.t $\mc X$ and $k$, the following holds.

If $\lambda \ge 3$ then for every $1 \le i \le k$, there exists a node $N_i$ in the tree $T$ such that $C_i^* = N_i$.
\end{theorem}

\begin{proof}
We will show that for $\lambda \ge 3$, $d$ satisfies the strong stability property. Hence, the claim will then follow from Theorem. 8 in \cite{balcan2008discriminative}. We know that $\mc C_{\mc X}^*$ satisfies $\lambda$-center sepration. Denote by $r := \max r(C_i^*)$. Let $A \subset C_i^*$ and $B \subseteq \mc C_j^*$. We will prove the claim in two steps. We will first show that $d_{min}(A, C_i^*\setminus A) \le r$ and then we will prove that $d_{min}(A, B) > r$.

Let $p \in A$ and $q \in C_i^* \setminus A$ be points which achieve the minimum distance between $A$ and $C_i^*\setminus A$. If $c_i \in A$ then $d(p, q) \le d(c_i, q) \le r$. If $c_i \in C_i^* \setminus A$ then $d(p, q) \le d(p, c_i) \le r$. Hence, $d_{min} (A, C_i^*\setminus A) \le r$.

Let $p \in A$ and $q \in B$ be points which achieve the minimum distance between $A$ and $B$. Using triangle inequality, we get that $d(p, q) \ge d(c_i, c_j) - d(p, c_i) - d(q, c_j) > 3r - r - r = r$.
\end{proof}

\subsubsection{Lower bound with no noise}
We will prove that for $\lambda \le 2$, finding any solution for $\lambda$-center separation is NP-Hard. \cite{reyzin2012data} proved that finding any solution for $\alpha$-center proximity is NP-Hard for $\alpha < 2$. Our proof is same as their proof. We show that the same reduction can be used to prove NP-Hardness for $\lambda$-center separation as well. We state the proof below for completeness.

\begin{theorem}
Given a clustering instance $\mc X$ drawn from a metric space $(\mb M, d)$ and the number of clusters $k$. For $\lambda < 2$, finding a clustering which satisfies $\lambda$-center separation is NP-Hard.
\end{theorem}
\begin{proof}
The proof is based on reducing an instance of Perfect dominating set promise problem (PDS-PP) to an instance of finding a clustering instance with $\lambda < 2$.

PDS-PP : Given a graph $G = (V, E)$. Given that all dominating sets of size $\le k$ are perfect. Find a dominating set of size $\le k$. \cite{reyzin2012data} showed that the PDS-PP is NP-Hard. 

For every $v \in V$ construct a point $x \in \mc X$ in the metric space $\mb M$. For any two points $x_1, x_2 \in \mc X$, define $d(x_1, x_2) = 1$ if an edge exists between the corresponding vertices $v_1$ and $v_2$. Else define $d(x_1, x_2) = 2$. Note that the function $d$ satisfies the triangle inequality. We will now show the following.

G has a dominating set of size $\le k \iff$ $\mc X$ has a $k$-clustering which satisfies $\lambda$-center separation for $1 \le \lambda < 2$.

$\Leftarrow$  Let $c_1, \ldots, c_k \in \mc X$ be the centers such that the corresponding clustering $\mc C = \{C_1, \ldots, C_k\}$ satisfies $\lambda$-center separation. Let $d_1, \ldots, d_k$ be the vertices in $V$ corresponding to the centers$c_1, \ldots, c_k$. Let $x \in C_i$ and $v \in V$ be the vertex corresponding to $x$. Now, $\lambda d(x, c_i) < d(c_i, c_j) \implies d(x, c_i) < d(c_i, c_j)$. Hence, $d(x, c_i) = 1$ and $d(c_i, c_j) = 2$. Thus, there exists an edge between $(v, d_i) \in E$. Hence, $d_1, \ldots, d_k$ forms a dominating set in $G$.

$\Rightarrow$ Let $D = \{d_1, \ldots, d_l\}$ be a dominating set in $G$. If $l < k$, we can add $k-l$ vertices to $D$ and they still form a dominating set. Hence, WLOG we can assume that $l = k$. Now, let $c_1, \ldots, c_k \in \mc X$ be the points corresponding to the vertices in $D$. Let $\mc C = \{C_1, \ldots, C_k\}$ be the clustering induced by $c_1, \ldots, c_k$. Since $|D| \le k$, we know that $D$ is perfect. Hence, $(d_i, d_j) \not\in E$. Hence, $d(c_i, c_j) = 2$. Now, let $v$ be any vertex which is dominated by $d_i$ and $x$ be the corresponding point in $\mc X$. Then, $d(x, c_i) = 1$. Hence, we get that $d(c_i, c_j) = 2 \max_p r(C_p)$. Hence, clustering $\mc C$ satisfies $\lambda$-center separation for $\lambda < 2$.
\end{proof}


\subsection{Center Separation in the presence of noise}

Similar to the framework in Section \ref{section:cp}, we do not work with any particular objective function. Given a set $\mc X$, we want to output a clustering $\mc C_{\mc X}$ of the set $\mc X$ which can capture  all clusterings of all the ``nice" subsets of $\mc X$. Nice subsets of $\mc X$ are those which satisfy $(\lambda, \eta)$-center $t$-min $r$-max proximity.


In this section, informally we try to address the following question. {\it Give that the non-noisy part of the data has $\lambda$-center separation, how much and what kind of noise can be tolerated by any hierarchical clustering algorithm?} We provide the following answers to this question.

\begin{itemize}
\item {\it Lower bound with arbitrary noise} - In Section \ref{section:lambdaLowerBoundArbitrary}, we show that if the number of noisy points $> \frac{3t}{2}$ and $\lambda \le 6$ then no algorithm can output a tree which captures all $\lambda$-center $t$-min proximal subsets of $\mc X$.
\item  {\it Positive result under sparse noise} - In Section \ref{section:lambdaPositiveResultSparseNoise}, we show that we can overcome this lower bound if we require that the noise is sparse. If $\lambda \ge 4$ and $\eta \ge 1$, then Alg.\ref{alg:lambdacs} outputs a clustering which captures all $(\alpha, \eta)$-center $t$-min $r$-max nice subsets of $\mc X$.
\item  {\it Lower bound under sparse noise} - In Section \ref{section:lambdaLowerBoundSparse}, we show that if $\lambda < 4$ and $\eta \le 1$, then no algorithm can output a tree which captures all $(\lambda, \eta)$-center $t$-min $r$-max nice subsets of $\mc X$.
\end{itemize} 


\subsubsection{Lower bound with arbitrary noise}
\label{section:lambdaLowerBoundArbitrary}
\begin{figure}
\ifdefined\COMPLETE
\else
\documentclass[12pt]{article}
\input{utils} %file containing all the used libraries
\begin{document}
\fi

\def\alph{$2+\sqrt{5}$}
\begin{tikzpicture}[scale=7]

	\coordinate (x) at ($ (-0.02,{sqrt(0)}) $);
	\coordinate (y) at (2.1,0);
	\draw[thick] (x) --(y);

	\node[label=270:$P_{t'}$] at (0.0,0) {$\bullet$};
	\node[label={[yshift=-0.2cm]\scriptsize $r$}] at (0.05,0) {};
	\node[label=below:{$P_2$}] at (0.1,0) {$\bullet$};
	\node[label={[yshift=-0.2cm]\scriptsize $r$}] at (0.15,0) {};
	\node[label=270:$P_{t'}$] at (0.2,0) {$\bullet$};
	\node[label={[yshift=-0.2cm]\scriptsize $r$}] at (0.25,0) {};
	\node[label=270:$P_2$] at (0.3,0) {$\bullet$};
	\node[label={[yshift=-0.2cm]\scriptsize $r$}] at (0.35,0) {};
	\node[label=270:$P_{t'}$] at (0.4,0) {$\bullet$};
	\node[label={[yshift=-0.2cm]\scriptsize $r$}] at (0.45,0) {};
	\node[label=270:$P_2$] at (0.5,0) {$\bullet$};
	\node[label={[yshift=-0.2cm]\scriptsize $r$}] at (0.55,0) {};
	\node[label=270:$P_{t'}$] at (0.6,0) {$\bullet$};	
	\node[label={[yshift=-0.2cm]\scriptsize $r$}] at (0.65,0) {};
	\node[label=270:$P_2$] at (0.7,0) {$\bullet$};	
	\node[label={[yshift=-0.2cm]\scriptsize $r$}] at (0.75,0) {};
	\node[label=270:$P_{t'}$](B) at (0.8,0) {$\bullet$};	
	\node at (1.89,0.03) {$\ldots\ldots\ldots$};	
	\node[label=270:$P_2$] at (0.9,0) {$\bullet$};	
	\node[label=270:$P_{t'}$] at (1.0,0) {$\bullet$};	
	\node[label={[yshift=-0.2cm]\scriptsize $r$}] at (0.85,0) {};
	\node[label=270:$P_2$] at (1.1,0) {$\bullet$};	
	\node[label={[yshift=-0.2cm]\scriptsize $r$}] at (0.95,0) {};
	\node[label=270:$P_{t'}$] at (1.2,0) {$\bullet$};	
	\node[label={[yshift=-0.2cm]\scriptsize $r$}] at (1.05,0) {};
	\node[label={[yshift=-0.2cm]\scriptsize $r$}] at (1.15,0) {};
	\node[label={[yshift=-0.2cm]\scriptsize $10r$}] at (1.37,0) {};
	\node[label=270:$B_3$](p3start) at (1.54,0) {$\bullet$};	
	\node[label={[yshift=-0.2cm]\scriptsize $r$}] at (1.64,0) {};
	\node[label=270:$B_4$](B) at (1.74,0) {$\bullet$};	
	\node at (1.89,0.03) {$\ldots\ldots\ldots$};	
	\node[label=270:$B_k$](p3end) at (2.04,0) {$\bullet$};	
	\node(A) at (1.79,0.1) {\small $k-2$};	
	
	\node[label=270:$B_1$] at (0.1,-0.1) {};	
	\node[label=270:$B_1'$] at (0.3,-0.1) {};	
	\node[label=270:$B_1''$] at (0.5,-0.1) {};	
	\node[label=270:$B_2$] at (0.7,-0.1) {};	
	\node[label=270:$B_2'$] at (0.9,-0.1) {};
	\node[label=270:$B_2''$] at (1.1,-0.1)
{};	
	\draw (0.1,0) circle (0.1);
	\draw[dashed, blue] (0.3,0) circle (0.1);
	\draw[dotted, green, thick] (0.5,0) circle (0.1);
	\draw (0.7,0) circle (0.1);
	\draw[dashed, blue] (0.9,0) circle (0.1);
	\draw[green, thick, dotted] (1.1,0) circle (0.1);
	\draw[decorate,decoration={brace,amplitude=5pt}]
	(1.54, 0.05) -- (2.04, 0.05);
	%\draw (A) -- (1.74, 0.06);
\end{tikzpicture}


\ifdefined\COMPLETE
\else
\end{document}
\fi

\caption{$\mc X \subseteq \mathbb{R}$ such that no algorithm can capture all the $\lambda$-separated $t$-min $r$-max nice subsets of $\mc X$}
\label{fig:nosparsealglambdacs}
\end{figure}


To prove the lower bound we first construct a set $\mc X$ . We then construct sets $\mc S, \mc S', \mc S'' \subset \mc X$ which are $\lambda$-center $t$-min $r$-max proximal. We then show that these sets are `incompatible'. That is, there doesn't exist a  hierarchical clustering tree that can capture all the clusterings of these sets that satisfy $\lambda$-center proximity. The set $\mc X$ is described in Fig. \ref{fig:nosparsealglambdacs}. We will prove that the clusterings $\mc C_{\mc S} = \{B_1, B_2, B_3, \ldots, B_k\}$, $\mc C_{\mc S'} = \{B_1', B_2', B_3, \ldots, B_k\}$ and $\mc C_{\mc S''} = \{B_1'', B_2'', B_3, \ldots, B_k\}$ are incompatible with one another.

\begin{theorem}
Given the number of clusters $k$ and parameters $\lambda$, $r$ and $t$. There exists a clustering instance $(\mc X , d)$ such that any clustering tree $\mc T$ of $\mc X$ has the following property. There exists $\mc S \subseteq \mc X$ and there exists $k$-clustering $\mc C_{\mc S} = \{S_1, \ldots, S_k\}$ which satisfies $\lambda$-center separation such that $t(\mc C_{\mc S}) = t$, $r(\mc C_{\mc S}) = r$ and the following holds.

If $\lambda < 6$ and $|\mc X\setminus \mc S|\ge \frac{3t}{2}+5$, then $\mc T$ doesn't capture $\mc C_{\mc S}$ .
\end{theorem}

\begin{proof}
Define $\mc X \subseteq \mathbb{R}$ as follows. Let $t' = \frac{t}{2}-1$. Let $O_1 = \cup_{i=0}^6 P_{t'}(2ir)$ and $O_2 = \cup_{i=0}^5 P_2(r+2ir)$. For $3\le i\le k$, define $B_i = P_t(22r+ir)$. Now, let $\mc X = \{O_1 \cup O_2 \cup B_3 \cup \ldots \cup B_k\}$. The set is also shown in Fig. \ref{fig:nosparsealg}.

Now, let $B_1 = \{P_{t'}(0) \thinspace\cup\thinspace  P_{2}(r) \thinspace\cup\thinspace P_{t'}(2r) \}$ and $B_2 = \{\ P_{t'}(6r) \thinspace\cup\thinspace P_{2}(7r)\thinspace\cup\thinspace P_{t'}(8r)\}$. Also, let $B_1' = \{P_{t'}(2r) \thinspace\cup\thinspace  P_{2}(3r) \thinspace\cup\thinspace P_{t'}(4r) \}$ and $B_2' = \{\ P_{t'}(8r) \thinspace\cup\thinspace P_{2}(9r)\thinspace\cup\thinspace P_{t'}(10r)\}$. Also, let $B_1'' = \{P_{t'}(4r) \thinspace\cup\thinspace  P_{2}(5r) \thinspace\cup\thinspace P_{t'}(6r) \}$ and $B_2'' = \{\ P_{t'}(10r) \thinspace\cup\thinspace P_{2}(11r)\thinspace\cup\thinspace P_{t'}(12r)\}$. 

Now, define $\mc S = \{B_1 \cup B_2 \cup B_3\ldots \cup B_k\} \subseteq \mc X$, $\mc S' = \{B_1' \cup B_2' \cup B_3\ldots \cup B_k\} \subseteq \mc X$ and $\mc S'' = \{B_1'' \cup B_2'' \cup B_3\ldots \cup B_k\} \subseteq \mc X$. It is easy to see that for $\lambda < 6$, all of $\mc S$, $\mc S'$ and $\mc S''$ are $\lambda$-separated $t$-min $r$-max nice and the clusterings $\mc C_{\mc S} = \{B_1, B_2, B_3, \ldots, B_k\}$, $\mc C_{\mc S'} = \{B_1', B_2', B_3, \ldots, B_k\}$ and $\mc C_{\mc S''} = \{B_1'', B_2'', B_3, \ldots, B_k\}$ satisfy $\lambda$-center separation.

$\mc C_{\mc X}$ such that $\mc C_{\mc X}|_{\mc S} = \mc C_{\mc S}$ and $\mc C_{\mc X}|_{\mc S'} = \mc C_{\mc S'}$ and $\mc C_{\mc X}|_{\mc S''} = \mc C_{\mc S''}$. For the sake of contradiction assume that this is the case. There exists cluster $C_1 \in \mc C_{\mc X}$ such that such that $B_1 \subseteq C_1$ and $C_1 \cap B_2 = \phi$. Similarly, there exists cluster $C_1' \in \mc C_{\mc X}$ such that such that $B_1' \subseteq C_1'$ and $C_1' \cap B_2' = \phi$ and there exists cluster $C_1'' \in \mc C_{\mc X}$ such that such that $B_1'' \subseteq C_1''$ and $C_1'' \cap B_2'' = \phi$. Now, $C_1, C_1', C_1'' \in \mc X$ and $C_1 \cap C_1' \neq \phi$ and $C_1'' \cap C_1' \neq \phi$. Hence, $C_1 = C_1' = C_1''$. Hence, $B_1 \cup B_1' \cup B_1'' \subseteq C_1$. Observe, that $B_1'' \cap B_2 \neq \phi$. Hence, $C_1\cap B_2 \neq \phi$. This is a contradiction.
\end{proof}

\subsubsection{Positive result under sparse noise}
\label{section:lambdaPositiveResultSparseNoise}
We are given a clustering instance $(\mc X, d)$ and parameters $r$ and $t$. Our goal is to output a clustering $\mc C_{\mc X}$ of $\mc X$ which has the following property. For every $k$, for every $\mc S \subseteq \mc X$ which is $(\lambda, \eta)$-separated $t$-min $r$-max nice, and every $k$-clustering $\mc C_{\mc S}$ which satisfies $(\lambda, \eta)$-center separation, the clustering $\mc C_{\mc X}$ restricted to $\mc S$ equals $\mc C_{\mc S}$. 

In the next section, we propose a clustering algorithm (Alg.\ref{alg:lambdacs}) and prove (Theorem.\ref{thm:lambdacsnoise}) that our algorithm indeed achieves the above mentioned goal (under certain assumptions on the parameters $\lambda$ and $\eta$). It is important to note that our algorithm only knows $\mc X$ and has no knowledge of the set $\mc S$. 

Intuitively, Alg.\ref{alg:lambdacs} works as follows. In the first phase, it constructs a list of balls which have radius atmost $r$ and contain atleast $t$ points. It then constructs a graph as follows. Each ball found in the first phase is represented by a vertex. If two balls have a `large' intersection then their is an edge between the corresponding vertices in the graph. We then find the connected components in the graph which correspond to the clustering of the original set $\mc X$. 


\begin{algorithm}[!ht]
	\KwIn{$(\mc X, d), t$ and $r$}
	\KwOut{A clustering $\mc C$ of the set $\mc X$.}
	
	\vspace{0.1in}\textbf{Phase 1}\\
	Let $\mc L$ denote the list of balls found so far. Initialize $\mc L$ to be the empty set. $\mc L = \phi$.
	
	Go over all pairs of points $p, q \in \mc X$ in increasing order of the distance $d(p, q)$. 		\begin{itemize}[noitemsep, nolistsep]
	\renewcommand\labelitemi{}
	\item Let $B := B(p, d(p, q))$. 
	\item If $|B| \ge t$ and $r(B) \le r$ then
		\begin{itemize}[noitemsep, nolistsep]
		\renewcommand\labelitemii{}
		\item $\mc L = \mc L \thinspace\cup B$
		\end{itemize}
	\end{itemize}
	Output the list of balls $\mc L = \{B_1, \ldots, B_l\}$ to the second phase of the algorithm.
	
	\vspace{0.1in}\textbf{Phase 2}\\
	Construct a graph $G = (V, E)$ as follows. $V = \{v_1, v_2, \ldots, v_l\}$. If $|B_i \cap B_j| \ge t/2$ then construct an edge between $v_i$ and $v_j$.
	
	Find connected components ($G_1, \ldots, G_{k}$) in the graph $G$. 
	
	Merge all the points in the same connected component together to get a clustering $\mc C = \{C_1, \ldots, C_k\}$ of the set $\mc X$.
	
	Assign $x \in \mc X \setminus \mc \cup_i B_i$ to the closest cluster $C_i$. That is, $i := \argmin\limits_{j\in [k]} \min\limits_{y \in C_j}d(x, y)$. Output $\mc C$. 
\caption{Alg. for $(\lambda, \eta)$-center separation with parameters $t$ and $r$}
\label{alg:lambdacs}
\end{algorithm}

\begin{theorem}
\label{thm:lambdacsnoise}
Given a clustering instance $(\mc X, d)$ and parameters $r$ and $t$. For every $k$, for every $\mc S \subseteq \mc X$ and for all $k$-clusterings $\mc C^*_{\mc S} = \{S_1^*, \ldots, S_k^*\}$ induced by centers $s_1, \ldots, s_k \in \mc X$ which satisfy $(4, 1)$-center separation w.r.t $\mc X, \mc S$ and $k$ such that $ t(\mc C_{\mc S}^*) = t$ and $r(\mc C_{\mc S}^*) = r$, the following holds.

Alg. \ref{alg:lambdacs} outputs a clustering $\mc C_{\mc X}$ such that $\mc C_{\mc X}|_{\mc S} = \mc C_{\mc S}^*$ that is the clustering retricted to the set $S$ is same as the clustering $\mc C_{\mc S}^*$.
\end{theorem}
\begin{proof}
Fix $\mc S \subseteq \mc X$. Let $\mc C_{\mc S}^* = \{S_1^*, \ldots, S_k^*\}$ be a clustering of $\mc S$ such that $\min |S_i^*| = t$ and $\max_i r(S_i^*) = r$. Also, denote by $r_i := r(S_i^*)$. Let $\mc C_{\mc X} = \{C_1, \ldots, C_k\}$ be the clustering outputed by the algorithm. Throughout the proof, we will sometimes refer to $S_i^*$ as the `good clusters' and the set $\mc S$ as the good set.

Let $\mc L = \{B_1, \ldots, B_l\}$ be the list of balls as outputed by Phase 1 of Alg. \ref{alg:lambdacs}. Let $G$ be the graph as constructed in Phase 2 of the algorithm. First observe that $s_i \in \mc X$ and $r_i \le r$. Hence, $B = B(s_i, r_i) \in \mc L$ and $B = S_i^*$. WLOG, denote this ball by $B^{(i)}$ and the corresponding vertex in the graph $G$ by $v^{(i)}$. We will prove the theorem by proving two key facts.  

\begin{enumerate}[nolistsep, noitemsep, label=\textbf{F.\arabic*}]
\renewcommand\labelitemi{$\diamond$}
\item \label{fact:lambda1} If two balls $B_{i1}$ and $B_{i2}$ intersect the same good cluster $S_i^*$ then the corresponding vertices $v_{i1}$ and $v_{i2}$ are connected by a path in $G$.
\item \label{fact:lambda2} If a ball $B_{i1}$ intersects a good cluster $S_i^*$ and another ball $B_{j1}$ intersects some other good cluster $S_j^*$ then the corresponding vertices $v_{i1}$ and $v_{j1}$ are disconnected in $G$.	
\end{enumerate}
Note that the theorem follows from these two facts. We now state both of these facts formally below and prove them.\\

\noindent\textit{\underline{Proof of Fact \ref{fact:lambda1}}}
\begin{claim}
\label{claim:lambda1}
Let $\mc L, G, B^{(i)}$ and $v^{(i)}$ be as defined above. Let balls $B_{i1}, B_{i2} \in \mc L$ be such that $B_{i1} \cap S_i^* \neq \phi$ and $B_{i2} \cap S_i^* \neq \phi$. Then there exists a path between vertices $v_{i1}$ and $v_{i2}$ in the graph $G$.
\end{claim}
\vspace{-0.1in} For the sake of contradiction, assume that $v_{i1}$ and $v^{(i)}$ are not connected by an edge in $G$. Hence, $|B_{i1} \cap B^{(i)}| < t/2 \implies |B_{i1} \setminus B^{(i)}| \ge t/2$. Now, observe that since $\lambda > 4$, for all $j \neq i$, $B_{i1} \cap S_j^* = \phi$. Thus, $B_{i1} \setminus B^{(i)} \subseteq \mc X \setminus \mc S$. Hence, we get that $|B_{i1} \cap \{\mc X \setminus \mc S\}| \ge t/2$. This contradicts the sparseness assumption. Hence, $v_{i1}$ is connected to $v^{(i)}$ by an edge in $G$. By symmetry, there exists an edge between $v_{i2}$ and $v^{(i)}$. Thus, $v_{i1}$ and $v_{i2}$ are connected in $G$.\\

\noindent\textit{\underline{Proof of Fact \ref{fact:lambda2}}}
\begin{claim}
Let the framework be as in Claim \ref{claim:lambda1}. Let $B_{i1} \in \mc L$ be such that $B_{i1} \cap S_i^* \neq \phi$ and $B_{j1}$ be such that $B_{j1} \cap S_j^* \neq \phi$. Then the vertices $v_{i1}$ and $v_{j1}$ are disconnected in $G$.
\end{claim}
\vspace{-0.1in} Now, we will show that if a ball $B_{i1}$ intersects $S_i^*$ and $B_{j1}$ intersects $S_j^*$ then the corresponding vertices $v_{i1}$ and $v_{j1}$ are disconnected in $G$. For the sake of contradiction, assume that $v_{i1}$ and $v_{j1}$ are connected by a path in $G$. Hence, there exists vertices $v_{i}$ and $v_{n}$ such that $v_i$ and $v_n$ are connected by an edge in $G$ and $B_i \cap S_i^* \neq \phi$ and $B_n \cap S_n^* \neq \phi$ for some $n \neq i$. $v_i$ and $v_n$ are connected, hence $|B_i \cap B_n| \ge t/2$. Now, $\lambda \ge 4$, thus $B_i \cap \{\mc S \setminus S_i^*\} = \phi$ and $B_n \cap \{\mc S\setminus S_n^*\} = \phi$. Thus, $B_i \cap B_n \subseteq \mc X \setminus \mc S$. Thus $|B_{i} \cap \{\mc X \setminus \mc S\}| \ge t/2$. This contradicts the sparseness assumption. Hence, we get that $v_{i1}$ and $v_{j1}$ are disconnected in $G$.
\end{proof}


\begin{theorem}
Given clustering instance $(\mc X, d)$ and parameters $r$ and $t$. Algorithm \ref{alg:lambdacs} runs in time which is polynomial in $|\mc X|$.
\end{theorem}

\begin{proof}
Let $n = |\mc X|$. Phase 1 of Alg. \ref{alg:lambdacs} runs in $O(n^2)$ time. Phase 2 gets as input a list of size $l$. Constructing the graph takes $O(l^2)$ time. Finding $k$ connected components also takes $O(l^2)$ time. The remaining steps take $O(n)$ time. Hence, the algorithm runs in $O(n^2 + l^2)$ time where $l = n/t$. Hence, the total running time is $O(n^2)$.
\end{proof}

\subsubsection{Lower bound under sparse noise}
\label{section:lambdaLowerBoundSparse}

To prove the lower bound we first construct a set $\mc X$ . We then construct sets $\mc S, \mc S' \subset \mc X$ which are $(\lambda, \eta)$-center $t$-min $r$-max proximal. We then show that these sets are `incompatible'. That is, there doesn't exist a  hierarchical clustering tree that can capture all the clusterings of these sets that satisfy $\lambda$-center proximity. The set $\mc X$ is described in Fig. \ref{fig:noalglambdacs}. We will prove that the clusterings $\mc C_{\mc S} = \{B_1, B_2, B_3, \ldots, B_k\}$ and $\mc C_{\mc S'} = \{B_1', B_2', B_3, \ldots, B_k\}$ are incompatible with one another.

\begin{figure}
\ifdefined\COMPLETE
\else
\documentclass[12pt]{article}
\input{utils} %file containing all the used libraries
\begin{document}
\fi

\def\alph{$2+\sqrt{5}$}
\begin{tikzpicture}[scale=7]

	\coordinate (x) at ($ (0.0,{sqrt(0)}) $);
	\coordinate (y) at (2.05,0);
	\draw[thick] (x) --(y);

	\node[label=270:$P_{t_2}$] at (0.0,0) {$\bullet$};
	\node[label={[yshift=-0.2cm]\scriptsize $r$}] at (0.075,0) {};
	\node[label=270:$P_1$] at (0.15,0) {$\bullet$};
	\node[label={[yshift=-0.2cm]\scriptsize $r$}] at (0.225,0) {};
	\node[label=below:{$P_{t_1}$}] at (0.3,0) {$\bullet$};
	\node[label={[yshift=-0.2cm]\scriptsize $r$}] at (0.375,0) {};
	\node[label=270:$P_1$] at (0.45,0) {$\bullet$};
	\node[label={[yshift=-0.2cm]\scriptsize $r$}] at (0.525,0) {};
	\node[label=270:$P_{t_2}$] at (0.6,0) {$\bullet$};
	\node[label={[yshift=-0.2cm]\scriptsize $r$}] at (0.675,0) {};
	\node[label=270:$P_1$] at (0.75,0) {$\bullet$};
	\node[label={[yshift=-0.2cm]\scriptsize $r$}] at (0.825,0) {};
	\node[label=270:$P_{t_1}$] at (0.9,0) {$\bullet$};
	\node[label={[yshift=-0.2cm]\scriptsize $r$}] at (0.975,0) {};
	\node[label=270:$P_1$] at (1.05,0) {$\bullet$};	
	\node[label={[yshift=-0.2cm]\scriptsize $r$}] at (1.125,0) {};
	\node[label=270:$P_{t_2}$] at (1.2,0) {$\bullet$};	
	\node[label={[yshift=-0.2cm]\scriptsize $10r$}] at (1.4,0) {};
	\node[label=270:$B_3$](p3start) at (1.55,0) {$\bullet$};	
	\node at (1.75,0.03) {$\ldots\ldots\ldots$};	
	\node[label=270:$B_k$](p3end) at (1.95,0) {$\bullet$};	
	\node(A) at (1.79,0.1) {\small $k-2$};	
	
	\node[label=270:$B_1$] at (0.15,-0.15) {};	
	\node[label=270:$B_1'$] at (0.45,-0.15) {};	
	\node[label=270:$B_2$] at (0.75,-0.15) {};	
	\node[label=270:$B_2'$] at (1.05,-0.15) {};
	
	\draw (0.15,0) circle (0.15) ;
	\draw[dashed] (0.45,0) circle (0.15);
	\draw (0.75,0) circle (0.15);
	\draw[dashed] (1.05, 0.0) circle (0.15) ;
	\draw[decorate,decoration={brace,amplitude=5pt}]
	(1.55, 0.05) -- (1.95, 0.05);
	%\draw (A) -- (1.74, 0.06);
\end{tikzpicture}


\ifdefined\COMPLETE
\else
\end{document}
\fi

\caption{$\mc X \subseteq \mathbb{R}$ such that no algorithm can capture all the $(\lambda, \eta)$-separated $t$-min $r$-max nice subsets of $\mc X$}
\label{fig:noalglambdacs}
\end{figure}

\begin{theorem}
Given the number of clusters $k$ and parameters $\lambda$, $\eta$, $r$ and $t$. There exists a clustering instance $(\mc X , d)$ such that any clustering tree $\mc T$ of $\mc X$ has the following property. There exists $\mc S \subseteq \mc X$ and there exists a $k$-clustering $\mc C_{\mc S} = \{S_1, \ldots, S_k\}$ which satisfies $(\lambda, \eta)$-center separation w.r.t $\mc X, \mc S$ and $k$ such that $t(\mc C_{\mc S}) = t$ and $r(\mc C_{\mc S}) = r$ and the following holds.

If $\lambda < 4$ and $\eta \le 1$, then $\mc T$ doesn't $\mc C_{\mc S}$.
\end{theorem}

\begin{proof}
Define $\mc X \subseteq \mathbb{R}$ as follows. Let $t_1 = \frac{t}{2}+1$ and $t_2 = \frac{t}{2}-2$. Let $O_1 = \cup_{i=0}^2 P_{t_2}(4ir)$ and $O_2 = \cup_{i=0}^1 P_{t_1}(2r+4ir)$ and $O_3 = \cup_{i=0}^3 P_{t_1}(r+2ir)$. For $3\le i\le k$, define $B_i = P_t(18r+ir)$. Now, let $\mc X = O_1 \cup O_2 \cup O_3 \cup B_3 \cup \ldots \cup B_k$. The set is also shown in Fig. \ref{fig:noalglambdacs}.

Now, let $B_1 = \{P_{t_2}(0) \thinspace\cup\thinspace  P_{1}(r) \thinspace\cup\thinspace P_{t_1}(2r) \}$ and $B_2 = \{\ P_{t_2}(4r) \thinspace\cup\thinspace P_{1}(5r)\thinspace\cup\thinspace P_{t_1}(6r)\}$. Also, let $B_1' = \{P_{t_1}(2r) \thinspace\cup\thinspace  P_{1}(3r) \thinspace\cup\thinspace P_{t_2}(4r) \}$ and $B_2' = \{\ P_{t_1}(6r) \thinspace\cup\thinspace P_{1}(7r)\thinspace\cup\thinspace P_{t_2}(8r)\}$.

Now, define $\mc S = \{B_1 \cup B_2 \cup B_3\ldots \cup B_k\} \subseteq \mc X$ and $\mc S' = \{B_1' \cup B_2' \cup B_3\ldots \cup B_k\} \subseteq \mc X$. It is easy to see that for $\lambda < 4$, both $\mc S$ and $\mc S'$ are $(\lambda, 1)$-separated $t$-min $r$-max nice and the clusterings $\mc C_{\mc S} = \{B_1, B_2, B_3, \ldots, B_k\}$ and $\mc C_{\mc S'} = \{\ B_1', B_2', B_3, \ldots, B_k\}$ satisfy $(\lambda, 1)$-center proximity. 

We will show that no algorithm can output a clustering $\mc C_{\mc X}$ such that $\mc C_{\mc X}|_{\mc S} = \mc C_{\mc S}$ and $\mc C_{\mc X}|_{\mc S'} = \mc C_{\mc S'}$. For the sake of contradiction assume that this is indeed the case. There exists cluster $C_1' \in \mc C_{\mc X}$ such that such that $B_1' \subseteq C_1'$ and $C_1' \cap B_2' = \phi$. Now, observe that $B_1'$ contains points from both $B_1$ and $B_2$. Hence, there exists a cluster $C' \in \mc C_{\mc X}$ such that $C' \cap B_1 \neq \phi$ and $C' \cap B_2 \neq \phi$. This is a contradiction.
\end{proof}

\section{Conclusion}
\todo

\bibliographystyle{alpha}
\bibliography{colt2016ClusteringNoise} 


\appendix
\section{Technical result}
\label{appendix:sectiontr}
\begin{figure}
\ifdefined\COMPLETE
\else
\documentclass[12pt]{article}
\input{utils} %file containing all the used libraries
\begin{document}
\fi

\def\alph{$2+\sqrt{5}$}
\begin{tikzpicture}[scale=7]

	\coordinate (x) at ($ (-0.02,{sqrt(0)}) $);
	\coordinate (y) at (2.1,0);
	\draw[thick] (x) --(y);

	\node[label=270:$P_1$] at (0.0,0) {$\bullet$};
	\node[label={[yshift=-0.2cm]\scriptsize $\alpha+1$}] at (0.1,0) {};
	\node[label=below:{$P_2$}] at (0.2,0) {$\bullet$};
	\node[label={[yshift=-0.2cm]\scriptsize $\alpha-1$}] at (0.26,0) {};
	\node[label=270:$P_1$] at (0.32,0) {$\bullet$};
	\node[label={[yshift=-0.2cm]\scriptsize $\alpha+1$}] at (0.42,0) {};
	\node[label=270:$P_2$] at (0.52,0) {$\bullet$};
	\node[label={[yshift=-0.2cm]\scriptsize $\alpha+1$}] at (0.62,0) {};
	\node[label=270:$P_1$] at (0.72,0) {$\bullet$};
	\node[label={[yshift=-0.2cm]\scriptsize $\alpha-1$}] at (0.78,0) {};
	\node[label=270:$P_2$] at (0.84,0) {$\bullet$};
	\node[label={[yshift=-0.2cm]\scriptsize $\alpha+1$}] at (0.94,0) {};
	\node[label=270:$P_1$] at (1.04,0) {$\bullet$};	
	\node[label={[yshift=-0.2cm]\scriptsize $10\alpha$}] at (1.29,0) {};
	\node[label=270:$B_3$](p3start) at (1.54,0) {$\bullet$};	
	\node[label={[yshift=-0.2cm]\scriptsize $\alpha$}] at (1.64,0) {};
	\node[label=270:$B_4$](B) at (1.74,0) {$\bullet$};	
	\node at (1.89,0.03) {$\ldots\ldots\ldots$};	
	\node[label=270:$B_1$] at (0.15,-0.15) {};	
	\node[label=270:$B_{12}$] at (0.5,-0.15) {};	
	\node[label=270:$B_2$] at (0.9,-0.15) {};
	\node[label=270:$B_k$](p3end) at (2.04,0) {$\bullet$};
	\node(A) at (1.79,0.1) {\small $k-2$};	
	
	\draw (0.16,0) circle (0.16cm);
	\draw (0.88,0) circle (0.16cm);
	\draw[dashed] (0.52,0) circle (0.2cm);
	\draw[decorate,decoration={brace,amplitude=5pt}]
	(1.54, 0.05) -- (2.04, 0.05);
	%\draw (A) -- (1.74, 0.06);
\end{tikzpicture}


\ifdefined\COMPLETE
\else
\end{document}
\fi

\caption{$\mc X \subseteq \mathbb{R}$ such that Alg. \ref{alg:alphacp} doesn't capture all the $(\alpha, \eta)$-center $t$-min proximal subsets of $\mc X$.}
\label{fig:algAlphacp}
\end{figure}

\begin{theorem}
\label{thm:algAlphacp}
Given the number of clusters $k$ and parameters $\alpha$, $\eta$, $t$ and $r$. There exists a clustering instance $(\mc X, d)$ such that Alg. \ref{alg:alphacp} outputs a tree $T$ with the following property. There exists $\mc S \subseteq \mc X$ and there exists clustering $\mc C_{\mc S} = \{S_1, \ldots, S_k\}$ which satisfies $(\alpha, \eta)$-center proximity w.r.t $\mc X, \mc S$ and $k$ such that $ t(\mc C_{\mc S}) = t$ and $r(\mc C_{\mc S}) = r$ and the following holds. 

If $\alpha < 2 + \sqrt 5$ and $\eta \le 1$ and then $T$ doesn't capture $\mc C_{\mc S}$.
\end{theorem}

\begin{proof}
Define $\mc X \subseteq \mathbb{R}$ as follows. 

Let $t' = \frac{t}{2}-1$. Let $O_1 = \{P_{t'}(0) \thinspace\cup\thinspace P_{t'}(2\alpha) \thinspace\cup\thinspace P_{t'}(4\alpha+2) \thinspace\cup\thinspace P_{t'}(6\alpha+2)\}$ and $O_2 = \{P_{2}(\alpha+1) \thinspace\cup\thinspace P_{2}(3\alpha+1) \thinspace\cup\thinspace P_{2}(5\alpha+1) \}$. For $3\le i\le k$, define $B_i = P_t (16\alpha+2+i \alpha )$. Now, let $\mc X = \{ O_1 \cup O_2 \cup B_3 \cup \ldots \cup B_k \}$. The set $\mc X$ is also shown in Fig. \ref{fig:algAlphacp}.

Now, let $B_1 = \{P_{t'}(0) \thinspace\cup\thinspace  P_{2}(\alpha+1) \thinspace\cup\thinspace P_{t'}(2\alpha) \}$ and $B_2 = \{\ P_{t'}(4\alpha+2) \thinspace\cup\thinspace P_{2}(5\alpha+1)\thinspace\cup\thinspace P_{t'}(6\alpha+2)\}$. Now, define $\mc S = \{B_1 \cup \ldots \cup B_k\} \subseteq \mc X$. It is easy to see that for $\alpha \le 2+\sqrt{5}$, $\mc S$ is $(\alpha, 1)$-center $t$-min nice and the clustering $\mc C_{\mc S} = \{B_1, \ldots, B_k\}$ satisfies $(\alpha, 1)$-center proximity. However, we will show that the tree outputted by Alg. \ref{alg:alphacp} doesn't contain a pruning such that the corresponding clustering respects $\mc C_{\mc S}$.

Denote by $B_{12} = \{P_{t'}(2\alpha) \cup P_{2}(3\alpha+1) \cup P_{t'}(4\alpha+2) \}$. Alg. \ref{alg:alphacp}, iterates through pairs of points in increasing order of their distance. Hence, it first merges points in $B_3$ to $B_k$ into $k-2$ clusters. Now, observe that $B_1, B_{12}$ and $B_{2}$ have the radius $\alpha + 1$ (as our centers must be part of the dataset $\mc X$). Hence, the algorithm can choose to merge points in $B_{12}$ into the same cluster before merging points from $B_1$ and $B_2$ together. In this case, there exists no pruning of the tree which respects the clustering $\mc C_{\mc S}$.
\end{proof}
\end{document}
